\begin{section}

1. Code hoisting in adjacent blocks in a diamond/triangle like structure.
2. O(n) Algorithm.
3. Hoisting equivalent computations in non-sibling BBs.
4. Hoisting across switch blocks.
5. Hoisting to improve PRE.
6. Hoisting to reduce critical path in OoO execution.
7. Ranking/Wavefront algorithm to hoist across switch blocks.

Global code motion.
 bb1: b = ...
 bb2: c = ...
 bb3: a = b + c (to be moved to b5)
 bb4: ... = a
 old-live-range = distance(bb1,bb3) + distance(bb2,bb3) + distance (bb3,bb4)
 new-live-rance = distance(bb1,bb5) + distance(bb2,bb5) + distance (bb5,bb4)
 distance(bbx, bby) = total instruction count in the path from bbx to bby
 If the new live-range is less than the old one it will be a good candidate
 for gcm. When both the ranges will be same:
  - It is a simple copy of kind a = b.
  - One of the operands is not a Instruction/Register.
  - In these cases we need to have some heuristic or we can ignore them.

 The decision for bb5 can be made by whether hoist/sink is beneficial.

 Loads can be moved early as long as there is load available in each branch
 or there is already a load/store from/to the same underlying object.
 Stores can be moved up if there is a store/load to/from the same object.
 The idea is to establish that the object has memory allocated to it.
 Moving load/store together may help with locality of reference.

 PRE via Global Value Numbering
 Remove redundant instructions which are:
  - Already available in the dominator.
  - Are in one of the sibling branches, i.e., the instruction is used at
    a point which shares a common dominator where all the use-operands
    are available.
 In some cases it is possible to generate redundancy by restructuring the code
 (Ref. Ras Bodik), but that I'll leave for the next iteration of this patch.


 Hoisting also reduces critical path length of execution in out of order machines (but not in sequential machines), by exposing ILP before the conditional where the instruction was hoisted. This feature has already been identified in the previous patch which improves ray-trace.
  
 Concerns:
 Safety of load instructions to be checked. We cannot hoist loads until all the paths in the parent BBs have the same load. This is to comply with C semantics, although I'm not sure about this yet.
 Finding a cost model in case optimistic hoisting does not give desired results.

It seems, code hoisting is beneficial in cases even if the computations are not redundant.
Say c = f(a, b) is an instruction. Hoisting up will reduce the liveness of registers a and b, but will only increase the liveness of c. So we gain 2:1 even when the computation is not redundant. For loads this is not the case because, load takes only one operand so the liveness remains the same, additionally hoisting too much loads can have adversely affect the cache behavior.
On the other hand sinking loads may improve the cache behavior, because we load as late as possible. But sinking computations may increase the live range.
We might want to to improve the code-hoisting to a global code motion algorithm.

\section{Hoisting scalars}
Scalars are the easiest to hoist because we do not have to analyze for mem-refs. As long as all the
operands are available (which in SSA form we do, or may be there are chained dependencies), the scalar can
be hoisted. Care must be taken in case of hoisting scalars too far, as that may increase register pressure
and result in spills. For example hoisting a scalar past a call. In that case the call may result in
save and restore of the register the scalar may be defined to. In our current implementation
we do not hoist scalars past a call. Another way to mitigate this problem would be to
reinstantiate (rematerialize) the computation after a call (may be as a different pass).

\section{Hoisting calls}
Calls can be hoisted by value-numbering all the arguments. For safety we have to check that no other side-effects are there
between call and the hoisting point, just like loads and stores.

\section{Hoisting stores}
We need to value number the address and the value being stored.

\section{Hoisting bottom up with optimistic approach}
Optimistic means collect the set of all potential redundancies and then discard the bad ones.

Bottom up:
Sort the basic blocks of potentially redundant computations based on DFS-in numbers.
Then start evaluating the safety checks for first two, if good then move to second one. That way we can even
partially hoist computations should we find that we cannot remove all the redundant computations.

We can also split the set of to partially hoist multiple times, if hoisting globally is not possible.

In the case when more than one computation from the same basic block shows up as redundant.
e.g.
BB1
load a;
....
store .;
....
load a;


BB2
load a

BB0 -> BB1
BB0 -> BB2

In this case both the loads from BB1 will show up as redundant (according to the current algorithm), but the complete
hoising will fail as the second load in BB1 is not safe to hoist. By following the bottom-up approach, the sorted list
of BBs will appear like this:

{ BB1, BB2(first load), BB2(second load) }

Now we will start evaluating the safety of BB1-l1 and BB2-l2 which will pass, then BB12-l1 and BB2-l2 will fail.
So we can revert back to BB12-l1 and hoist them to BB0. Doing the top down approach and finding the partial set
to hoist instructions partially would result in a combinatorial explosion.

GCC: 23286 has interesting test cases.



\section{Enabling LICM by code-hoisting}
loop:
if (a)
  i1;
else
  i1;

Here i1 is redundant and can be hoisted out of loop. But the LICM will not do because it does not reason about instructions in the conditional.
By code-hoisting i1 will be hoisted out of conditional which will benefit the LICM.

\section{Improving the Inliner heuristics}
Reducing the number of instructions also reduces the inline cost and hence improves the inliner heuristics.

\section{Improving vectorization}
It will also benefit vectorizer by reducing the number
of basic blocks, should all the instructions in conditional get hoisted.


\end{section}


