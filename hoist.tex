\documentclass{sig-alternate}
\usepackage{comment}
\usepackage{amsmath}
\usepackage{hyperref}
\usepackage{graphicx}
\usepackage{amssymb}
\usepackage{graphviz}
\usepackage{auto-pst-pdf}
\usepackage{etoolbox}
\usepackage{flushend}
\usepackage{needspace}

\makeatletter
\preto{\@verbatim}{\topsep=1pt \partopsep=0pt}
\makeatother

\pagenumbering{arabic}

\begin{document}
\def \SCoP {SCoP}
\def \GCC {GCC}
\def \LLVM {LLVM}
\def \SESE {SESE}
\def \CFG {CFG}
\def \SSA {SSA}
\def \scev {scev}

\special{papersize=8.5in,11in}
\setlength{\pdfpageheight}{\paperheight}
\setlength{\pdfpagewidth}{\paperwidth}

\title{GVN-Hoist: Hoisting Computations from Branches}

\toappear{
   \hrule \vspace{5pt}
   LLVM-dev 2016
}
\numberofauthors{2}

\author{
\alignauthor
Aditya Kumar\\
       \affaddr{Samsung Austin R\&D Center}\\
       \email{aditya.k7@samsung.com}
\alignauthor
Sebastian Pop\\
       \affaddr{Samsung Austin R\&D Center}\\
       \email{s.pop@samsung.com}
%\and
%\alignauthor
%Daniel Berlin\\
%       \affaddr{Google Inc.}\\
%       \email{dberlin@dberlin.org}
}

\maketitle
\begin{abstract}
  Code-hoisting identifies identical computations across the program and hoists
  them to a common dominator so as to save code size.  Although the main goal of
  code-hoisting is not to remove redundancies: it effectively exposes
  redundancies and enables other passes like LICM to remove more redundancies.
  The main goal of code-hoisting is to reduce code size with the added benefit
  of exposing more instruction level parallelism and reduced register pressure.

  We present a code hoisting pass that we implemented in LLVM. It is based on
  Global Value Numbering infrastructure available in LLVM. The experimental
  results show an average of 2.5\% savings in code size, although the code size
  increases in many cases because it enables more inlining. This is an
  optimistic algorithm in the sense that we consider all identical computations
  in a function as potential candidates to be hoisted. We make an extra effort
  to hoist candidates by partitioning the potential candidates in a way to
  enable partial hoisting in case common hoisting points for all the candidates
  cannot be found. We also formalize cases when register pressure will reduce as
  a result of hoisting.
\end{abstract}

\section{Introduction}

Compiler techniques to remove redundant computations are composed of an analysis
phase that detects identical computations in the program and a transformation
phase that reduces the number of run-time computations.  Classical scalar
optimizations like CSE \cite{dragonbook} work very well on single basic blocks.
When it comes to detect redundancies across basic blocks these techniques fall
short: more complex passes like GCSE and PRE have been designed to handle these
cases based on dataflow analysis \cite{morel1979global}.  At first these
techniques were described in the classical data-flow analysis framework, and
later the use of the SSA representation lowered the cost in terms of compilation
time \cite{briggs1994effective,chow1997new,kennedy1999partial} and brought these
techniques in the main stream: nowadays SSA based PRE is available in every
industrial compiler.

This paper describes code-hoisting, a technique that uses the information
computed for PRE to detect identical computations and has a transformation phase
whose goal differs from PRE: it removes identical computations from different
branches of execution.  These identical computations in different branches of
execution are not redundant computations at run-time and the number of run-time
computations is not reduced. Code-hoisting is not a redundancy elimination pass,
and thus it has different cost function and heuristics than PRE or CSE.
Code-hoisting is also different from global code scheduling
\cite{dragonbook,click1995global} in the sense that code-hoisting will only
hoist computations when there are identical computations sharing a common
dominator. The goals of code hoisting are:

\begin{itemize}
\item to reduce the code size of the program;
\item to improve function inlining heuristics: functions become cheaper to
  inline by reducing their code size;
\item to expose more instruction level parallelism: by hoisting identical
  computations to be executed earlier, instruction schedulers can move heavy
  computations earlier in order to avoid pipeline bubbles;
\item to help out-of-order processors with speculative execution of branches: by
  hoisting expressions out of branches, code hoisting can effectively reduce the
  amount of code to be speculatively executed and can reduce the critical path;
\item to reduce register pressure: by moving computations closer to the
  definitions of their operands;
\item to improve passes that do not work well with branches:
  \begin{itemize}
  \item to improve loop vectorization by reducing a loop with control flow to a
    loop with a single BB, should all the instructions in a conditional get
    hoisted and sinked;
  \item to enable more loop invariant code motion (LICM): as LICM does not
    reason about instructions in the context of loops with conditional branches,
    code-hoisting is needed to move instructions out of conditional expressions
    and expose them to LICM.
  \end{itemize}
\end{itemize}

The main contributions of this paper are:
\begin{itemize}
\item a new algorithm to hoist computations from branches,
\item cost models to reduce live-range and reduce spills,
\item performance evaluation of the implementation in LLVM,
\item comparison against other algorithms for code hoisting.
\end{itemize}

\section{Related Work}

There are a lot of bug reports in GCC and LLVM bugzillas
\cite{GCCCodeHoistingBugs,LLVMCodeHoistingBugs}, showing the interest in having
a more powerful code hoist transform.  The current LLVM implementation of code
hoisting in SimplifyCFG.cpp $HoistThenElseCodeToIf()$ is very limited to
hoisting from identical basic blocks: the instructions of two sibling basic
blocks are read in the same time, and all the instructions of the blocks are
hoisted to the common parent block as long as the compiler is able to prove that
the instructions are equivalent.  This implementation does not allow for an easy
extension: first in terms of compilation time overhead the implementation would
become quadratic in number of instructions to bisimulate, and second the
equivalence of instructions currently uses $$I1->isIdenticalToWhenDefined(I2)$$
would need to be rewritten to be more general, leading to using a mechanism
similar to the idea described in this paper based on GVN.

Rosen et al. explain moving computations successors to remove redundancies
\cite{rosen1988global}. Their algorithm iterates on computations of same rank
and move the code with identical computations from the sibling branch. Also
there is no notion of partially hoisting the computations so their approach may
result in missing many hoisting opportunities.

Dhamdhere \cite{dhamdhere1988fast}, Muchnick \cite{steven1997advanced} mention
code hoisting in a data flow framework. A list of Very Busy Expressions (VBE)
are computed which are hoisted in a basic block where the expression is
anticipable (all the operands are available). This algorithm would hoist as far
as possible without regarding the impact on register pressure and as such a cost
model will be required.  Also the description of VBE is based on the classic
dataflow model and an adaptation to a sparse SSA representation is required.

GCC recently got code-hoisting which is implemented as part of GVN-PRE: it uses
the set of ANTIC and AVAIL value expressions computed for PRE.  The algorithm
hoists top down in a predecessor where the value is in ANTIC\_IN.  It uses
ANTIC[B] to know what expressions will be computed on every path from B to exit,
and can be computed in B: that is the safety condition.  It uses AVAIL[B] to
subtract out those values already being computed (because they are already
available): this is a redundancy condition.  The cost function is: for each
hoist candidate, if all successors of B are dominated by B, then we know
insertion into B will eliminate all the remaining computations.  It then checks
to see if at least one successor of B has the value available.  This avoids
hoisting it way up the chain to ANTIC.  It also checks to ensure that B has
multiple successors, since hoisting in a straight line is pointless.  The
algorithm continues on down the dominator tree, iterating with PRE until no more
changes.  One advantage of GCC implementation is that it works in sync with the
GVN-PRE such that when new hoisting opportunities are created by GVN-PRE,
code-hoisting will hoist them.

\newpage

\section{Code hoisting}

The algorithm for code hoisting uses several common representations of the
program that we shortly describe below:
\begin{itemize}
\item Control Flow Graph (CFG) and the Dominance (DOM) and Post-Dominance (PDOM)
  relations \cite{dragonbook};
\item Single Entry Single Exit (SESE) \cite{sese} and Single Entry Multiple Exit
  (SEME) regions;
\item Static Single Assignment (SSA) \cite{cytron};
\item Global Value Numbering (GVN) \cite{rosen1988global,click1995global}: to
  identify similar computations compilers use GVN.  Each expression is given a
  unique number and the expressions that the compiler can prove identical are
  given the same number;
\item Memory SSA \cite{novillo2007memory}: memory operations that the compiler
  is able to prove in dependence are linked through use-def chains.
\end{itemize}

The code-hoisting pass can be divided into the following steps that we will
describe in the rest of this section:
\begin{itemize}
\item find candidates: instructions computing same values,
\item compute a point in the program where it is both legal and profitable to
  move the code,
\item transform the code.
\end{itemize}

\subsection{Finding candidates to hoist}
The first step is to find a set of instructions that perform identical
computations: this is performed by a linear scan of all instructions of the
program and classifying all instructions by their value given by GVN.

The current implementation of GVN in LLVM has some limitations when it comes to
loads and stores so we compute the GVN of loads and stores separately.  Our
short-time solution to value number loads is to hash the address from where the
value is to be loaded. For stores, we value number the address as well as the
value to be stored at that address.

Another limitation of the current GVN implementation in LLVM is that the
instructions dependent on the loads will not get numbered correctly, and so
after hoisting all candidates we need to rerun the GVN analysis in order to
discover new candidates now available after having hoisted load instructions.
This limitation should be addressed in a new implementation of the GVN based on
MemorySSA, that would better account for equivalent loads and their dependent
instructions.

The process of computing GVN can be on-demand (as we come across an instruction)
or, precomputed (computing GVN of all the instructions beforehand). Which
process to choose is determined by the scope of code-hoisting we want to
perform. In a pessimistic approach, as described in
Section~\ref{subsec:pessimistic}, we want to hoist a limited set of instructions
from the sibling branches as we iterate the DFS tree bottom-up, it is sufficient
to compute GVN values on-demand. Whereas, in the optimistic approach, as
described in Section~\ref{subsec:optimistic}, we want to hoist as many
instructions as possible, and it would require GVN values to be precomputed.

Once the instructions have been classified in equivalence classes, we compute
for each group of equivalent instructions a point in the program that is both
legal and profitable for the instructions to be moved to.

\subsection{Legality check}
\label{subsec:legality}
Since the equality of candidates is purely based on the value they compute, we
need to establish if hoisting them to a common dominator would be feasible. Once
a common dominator is found, we check whether all the use-operands of the set of
instructions are available at that position. In some cases when the operands are not
available, it is possible to reinstantiate (remateralize) the use-operands, thus
passing the legality check.

Subsequently, it is checked that the side-effects of the computations does not
intersect with any side-effects between the instructions to be hoisted and their
hoisting point. For memory operations like loads, stores, calls etc., it is also
required to check that all the paths from the hoisting point to the end of the
function should execute the exact instruction, in order to guarantee
correctness.

Moreover, hoisting of memory operations is tricky on paths which have indirect
branch targets e.g., landing pad, case statements, goto labels etc., because it
becomes difficult to prove that all the paths from hoisting point to the end of
the function would execute the instruction. In our current implementation we
discard hoisting through such paths.

In the optimistic approach, described in Section~\ref{subsec:optimistic}, it is
possible that a common hoisting point of all the instructions is either too far
away, or not legally possible. In these cases, it is still possible to
`partially' hoist a subset of instructions by splitting the set of candidates
and finding a closer hoisting point for each subset. For more details see
Section~\ref{subsec:partition}.

\subsubsection{Legality of hoisting scalars}
Scalars are the easiest to hoist because we do not have to analyze them for
aliasing memory references. As long as all the operands are available (or can be
made available by rematerialization), the scalar computation can be hoisted.

\subsubsection{Legality of hoisting loads}
The availability of operand to the load (an address) is checked at the hoisting
point. If that is not available we try to rematerialize the addresss if
possible.  Along the path, from current position of the load instruction
backwards on the control flow to the hoisting point, we check whether there are
writes to memory that may alias with the load, in which case we discard the
candidate.

\subsubsection{Legality of hoisting stores}
For stores, we check the dependency requirements similar to the hoisting of
loads. We check that the operands of the store instruction are available at the
hoisting point, that there are no aliasing loads or store along the path from
the current position to the hoisting point.

\subsubsection{Legality of hoisting calls}
Call instructions can be divided into three categories: those calls equivalent
to purely scalar computations, calls reading from memory, and most of the time,
without further information, calls have to be classified as writing to memory,
that is the most restrictive form.  Each category of call instructions will be
handled as described for scalar, load, and store instructions.

\subsection{Profitability check}
\label{subsec:profitability}
After the legality checks have passed, we check for profitability of hoisting.
That takes into account the impact code-hoisting would have on various
parameters that affect runtime performance e.g., impact on live-range, gain in
the code size.  We have established a set of cost models described in Section~\ref{sec:cost-models}
for each parameter and tuned them for performance against representative
benchmarks.

\subsubsection{Profitability of hoisting scalars}
Since scalars are the majority of instructions which are hoisted, we pay special
attention in case of hoisting scalars too far, as that may increase register
pressure and result in spills. For example hoisting a scalar past a call, as
described in Section~\ref{cost:across-calls}.  In our current implementation we
hoist scalars past a call only when optimizing for code-side (-Os). Ideally, a
later stage of live-range splitting pass should split the live-ranges for
optimal performance, however, that is not the case with llvm as we have found
regressions when scalars are hoisted too far, as in
Section~\ref{sec:cost-models}. Another way to mitigate this problem is be to
reinstantiate (rematerialize) the computation after a call (may be as a
different optimization pass).

\subsubsection{Profitability of hoisting loads}
A load instruction introduces a register where the value loaded will be kept,
the register pressure increases by one (unless the operand to load becomes dead
at the load). On the other hand, loading a value early will reduce the stall
during execution should the value is not in the cache. We generally prefer to
hoist load except the hoisting point is too far (this distance is computed by
looking at the experimental results of representative benchmarks, see
Section~\ref{sec:experimental-results}).

\subsubsection{Profitability of hoisting stores}
Since stores do not increase the live-range of any registers, and in some cases
it ends the liveness of registers, we hoist all the stores.

\subsubsection{Profitability of hoisting calls}
Currently we hoist all the calls that are suitable candidates for hoisting.

\subsection{Code generation}
Once all the legality and profitability checks are satisfied for a set of
identical instructions, they are suitable candidates for hoisting. A copy of the
computation is inserted at the hoisting point along with any instructions which
needed to be rematerialized. Thereafter, all the computations made redundant by
the new copy are removed, and the SSA form is restored by updating the
intermediate representation (IR) to reflect the changes.

After one iteration of algorithm runs through the entire function, it creates
more oppportunities for \emph{higher ranked} computations
\cite{rosen1988global}. Currently, this is a limitation of the GVN analysis
pass, and so we rerun the code-hoisting algorithm until there are no more
instructions left to be hoisted.  Obviously, this is not the most optimal
approach and can be improved by ranking the computations \cite{rosen1988global},
or by improving the GVN analysis to correctly number loads and dependent
instructions.

Finally after the transform is done, we verify a set of post-conditions to
establish that program invariants are maintained: e.g., consistency of
use-defs, and SSA semantics.

\newpage

\subsection{Illustrative Example}
Code hoisting can also reduce the critical path length of execution in out of
order machines. As more instructios are available at the hoisting point, the
hardware has more instructions to reorder. Following example illustrates how
hoisting can improve performance by exposing more ILP.

\begin{verbatim}
float foo(float d, float min, float max, float a)
{
  float tmin, tmax, inv;

  inv = 1.0f / d;
  if (inv >= 0) {
    tmin = (min - a) * inv;
    tmax = (max - a) * inv;
  } else {
    tmin = (max - a) * inv;
    tmax = (min - a) * inv;
  }
  return tmax + tmin;
}
\end{verbatim}

In this program the computations of tmax and tmin are identical to the
computations of tmin and tmax of sibling branch respectively. Both tmax and tmin
depends on inv which depends on a division operation which is generally more
expensive than the addition, subtraction and multiplication operations. The
total latency of computation across each branch is:
$O(div) + 2(O(sub) + O(mul))$
Or, for out of order processors with two add units and two multiply units:
$O(div) + O(sub) + O(mul)$

Now if the computation of tmax and tmin are hoisted outside the
conditionals, the C code version would look like this:
\begin{verbatim}
float foo(float d, float min, float max, float a)
{
  float tmin, tmax, tmin1, tmax1, inv;

  tmin1 = (min - a);
  tmax1 = (max - a);

  inv = 1.0f / d;
  tmin1 = tmin1 * inv;
  tmax1 = tmax1 * inv;

  if (inv >= 0) {
    tmin = tmin1;
    tmax = tmax1;
  } else {
    tmin = tmax1;
    tmax = tmin1;
  }

  return tmax + tmin;
}

\end{verbatim}

In this code the two subtractions and the division operations can be executed in
parallel because there are no dependencies among them. So the total number of
cycles will be $max(O(div), O(sub)) + O(mul) = O(div) + O(mul)$; since $O(div)$ is
usually much greater than $O(sub)$ \cite{x86,aarch64}

Of course, a partial redundancy elimination pass could just remove the entire
if-block because final operation is an addition (asociative under fast math).

\begin{verbatim}
float foo(float d, float min, float max, float a)
{
  float tmin1, tmax1, inv;

  tmin1 = (min - a);
  tmax1 = (max - a);

  inv = 1.0f / d;
  tmin1 = tmin1 * inv;
  tmax1 = tmax1 * inv;

  return tmax1 + tmin1;
}
\end{verbatim}


GCC: 23286 has interesting test cases.

\section{Code hoisting policies}
The amount of hoisting depends on whether we collect GVN of instructions
before finding candidates (optimistic) or, on-demand (pessimistic). It also
depends on the generality of the GVN algorithm, however, that analysis is beyond
the scope of this paper.

\subsection{Hoisting bottom up with optimistic approach}
\label{subsec:optimistic}
In this approach, the goal is to maximise the total number of hoistings in the
entire function.  This algorithm is very useful when optimizing for code-size.
We collect the GVN of all the instructions in the function and iterate on the
list of instructions having identical GVNs. After that we find the common
dominator dominating all such identical computations and perform legality
checks, as described in Section~\ref{subsec:legality}. Often times it is not
possible to hoist all the instructions to one common dominator, due to legality
constraints e.g., intersecting side-effects or, profitability constraints e.g.,
hoisting point too far. In those cases, this algorithmn would partition the list
of identical instructions into subsets which can be partially hoisted to their
respective common dominators. The partition algorithm is described as follows:

\subsubsection{Partition the list of hoisting candidates to maximize hoisting}
\label{subsec:partition}
In order to hoist a subset of identical instructions, we partition the list of
all candidates in a way to maximise the total number of hoistings.  By sorting
the list of all the candidates in the increasing order of their depth first
search discovery time stamp \cite{clrs} (DFSIn numbers), we make sure that
candidates closer in the list have their common dominator nearby in cases when
there are no fully redundant instructions. Essentially if,

\begin{verbatim}
// B1 != B2 != B3
BasicBlock B1, B2, B3
DFSIn(B1) = depth first discovery time stamp
DFSIn(B1,B2) = |DFSIn(B1) - DFSIn(B2)|
DFSIn(B1,B3) = |DFSIn(B1) - DFSIn(B3)|

Depth(B1) = depth of tree from the root node.
// When B1 dominates B2
BBDist(B1, B2) = |Depth(B1) - Depth(B2)|

NCD(B1, B2) = nearest common dominator of B1 and B2
BBDist(B1,B2) = BBDist(B1, NCD(B1, B2))
BBDist(B1,B3) = BBDist(B1, NCD(B1, B3))

Then,
DFSIn(B1,B2) < DFSIn(B1,B3) implies
 BBDist(B1,B2) <= BBDist(B1,B3)
\end{verbatim}

Same property would hold for instructions I1, I2, I3 provided they are not fully
redundant i.e., none of the basic blocks containing I1, I2 and I3 respectively
dominate the other:

\begin{verbatim}
Instruction I1, I2, I3
// I1, I2, I3 are not fully redundant
// B1, B2, B3 contains I1, I2, I3 respectively
DFSIn(I1) = DFSIn(B1)
BBDist(I1, I2) = BBDist(B1, B2)
DFSIn(I1) = DFSIn(B1)
DFSIn(I1,I3) = DFSIn(B1, B3)
NCD(I1, I2) = NCD(B1, B2)

Then,
 DFSIn(I1,I2) < DFSIn(I1,I3) implies
  BBDist(I1,I2) <= BBDist(I1,I3)
\end{verbatim}

It says that if I1 is closer to I2 than I3 in the list of candidates sorted by
DFSIn numbers, then the nearest common dominator of I1 and I2 will be closer (in
terms of basic block distance) to I1, than nearest common dominator of I1 and
I3.

In the presence of fully redundant computations in the list of hoistable
candidates this equation may not hold. For example, if there is another
instruction I4 which is dominated by I1, but I4 is present in a basic block
which is farther than DFSIn(I1, I2) i.e., $DFSIn(I1,I2) < DFSIn(I1,I4)$;
however, $BBDist(I1,I4) = 0 < BBDist(I1,I2) >=1$.

So by sorting candidates w.r.t. their DFSIn numbers:
\begin{enumerate}
\item would make fewer checks for legality and profitability.
\item hoisting point will be closer, so the intersection of live-range of the
  instruction with other instructions will be minimal.
\end{enumerate}

In our current implementation we keep as many candidates in one set as possible
(greedy approach). We split the list at a point where the legality checks fail
to hoist subset of candidates which are legal to hoist and then start finding
new hoisting point for the remaining ones. So there are two limitations of the
current implementation of the partition algorithm: First, sorting by DFSIn
numbers does not give the desired order when there are fully redundant
instructions; Second, it lacks cost model for partitioning in order to
preserve/improve performance, see Section~\ref{sec:future-work}.

\subsection{Hoisting bottom up with pessimistic approach}
\label{subsec:pessimistic}
In the pessimistic approach, the basic blocks are traversed in the inverse
depth-first order, computing the GVN of instructions as they come by. The GVN of
sibling branches are compared for equality. Once such a candidate is found, it
is hoisted in the common dominator. All the leaf nodes are visited before the
non-leaf nodes (bottom-up) because instructions are hoisted upwards.

The pessimistic algorithm is fast, results in fewer spills but hoists very less
instructions. We have implemented optimistic approach because it is more general
and can be tuned down to closely mimic pessimistic approach by changing a flag.

\subsection{Time complexity of algorithm}
The complexity of code hoisting is linear in number of instructions that could
be hoisted in the program, matching the complexity of PRE on SSA form.  The
analysis phase is based on the Global Value Numbering (GVN), the same analysis
used for PRE, followed by the computation of a partition of identical
expressions to be hoisted in a same location to guarantee safety properties and
program performance, and followed by a simple code generation that adds the
identified instruction in the hoisting point and removes all the now redundant
expressions.

\section{Cost models}
\label{sec:cost-models}
Similar to any compiler optimization pass, there are several cost functions that
are deployed to tune for optimal combination of performance and code-size.
Since this is mostly a code-size optimization pass, the goal is to not regress
in performance across popular benchmarks at the same time reduce code size as
much as possible. Following are the cost models which are implemented:

\subsection{Reduce register pressure}
\label{hoist:reg-pressure}
Following example explains how code hoisting can actually reduce the register
pressure.  Consider the following example where the labels prefixed with 'P'
represent the position of instruction in a basic block (names prefixed with 'B').

\begin{verbatim}
B0: P0: b = 1
    goto B1

B1: P1: c = 2
    goto B2

B2: P2: if c is true then goto B3 else goto B4

B3: P3: a0 = b + c
    goto B5

B4: P4: a1 = b + c
    goto B5

B5: P5: d = phi {a0(B3), a1(B4)}

If we measure D(Px,Py) as total instruction
count in the path from the position of Px to Py

live-range(a0) = D(P0,P3) + D(P1,P3) + D(P3,P5)
               = 6 + 4 + 2 = 12
live-range(a1) = D(P0,P4) + D(P1,P4) + D(P4,P5)
               = 6 + 4 + 2 = 12

old-live-range = max(live-range(a0),live-range(a1))
               = 12

After hoisting a0 and a1 are removed and a copy
of a0 as a01 is placed in B2 just before P2.

live-range(a01) = D(P0,P2-1) + D(P1,P2-1) + D(P2 -1,P5)
                = 4 + 3 + 3 = 10
new-live-range = live-range(a01)
\end{verbatim}

If the new live-range is less than the old one it will be a good candidate for
hoisting. The live ranges can remain same as well if there is only one operand
on the right hand side e.g., an assignment operation or, one of the operands is
not a register. That means hoisting upwards will decrease the live-range of its
use but increase the live-range of its definition.

In a special case where the instruction to be hoisted has the last use of its
operands then the code hoisting will always reduce the register pressure if it
has two register operands because the gain in live-range will be in the ratio of
2:1. Based on the above formulae we can also deduce that, as long as there is
one register operand in the right hand side with its last use, code hoisting
will either decrease or preserve the register pressure.

\subsection{In the presence of calls}
\label{cost:across-calls}
Hoisting scalars across calls is tricky because it can increase the number of
spills. During the frame lowering of calls, the argument registers, in general,
the caller saved registers are saved because they might be modified by the
callee and after the call they are restored \cite{frame-lowering}. So before the
call, the register pressure is high because the number of available registers
are reduced by the number of caller saved registers. In that situation if a
computation is hoisted across the call, that would increase the total number of
registers required by one, thus contributing to the register pressure. However,
in the special case discussed in Section~\ref{hoist:reg-pressure}, it will be
okay to hoist because the register pressure would not decrease.

Hoisting loads/stores across calls also require precise analysis of all the
memory addresses accessed by the call. Our implementation being an
intraprocedural pass, the analysis is very conservative. In the presence of pure
calls, loads can be hoisted but stores can't. Also, if the call throws
exceptions, or it it may not return, memory references cannot be hoisted.

\subsection{Hoisting too far away}
If there are several instructions in between the hoisting point and the
instruction to be hoisted, the instruction to be hoisted crosses several
instructions while hoisting, that means we are adding one register to all the
live-ranges spanning the instructions. That could result in spills. In the
current implementation we choose to hoist if the number of instructions crossed
is below a threshold. Ideally, it should be okay to hoist all the instructions
and a later a live-range-splitting \cite{cooper1998live} pass should make the
right decision of rematerializing the instruction should it be beneficial to do
so. But the current live-range splitting pass of llvm is not making the optimial
decision and we have found spills if the threshold is exceeded. The threshold
was computed as a result of tuning the llvm testsuite \cite{llvm-nightly} and
spec benchmark \cite{Henning2000}.

Also, hoisting a load increases the register pressure by one across all the
instructions which the load would cross. That could result in spills later in
the register allocation.

However, in the special case discussed in Section~\ref{hoist:reg-pressure}, it
will be okay to hoist because the register pressure would not decrease.


\section{Experimental Evaluation}
\label{sec:experimental-results}
We ran llvm-testsuite (trunk:d87471f8) with the patch (trunk:86940146) and the
results are listed in Table~\ref{tab:hoist-results}. The table lists the number
of scalars, loads, stores and calls hoisted as well as removed. For each
category, the number of instructions removed is greater or equal to the number
of instructions hoisted because each hoisting is performed only when at least
one identical computation is found.

Loads are hoisted the most followed by scalars, stores and calls in decreasing
order.  This was the common trend in all our experiments. One reason why loads
are hoisted the most is the early execution of this pass (before mem2reg) in the
llvm pass pipeline. Passes like mem2reg, instcombine might actually remove
those loads so this order may change should this pass be scheduled later.

\begin{table}[h!]
  \begin{center}
    \begin{tabular}{|l|c|}
      \hline
      Metric               & Number\\\hline
      Scalars hoisted      & 6791  \\\hline
      Scalars removed      & 9696  \\\hline
      Loads hoisted        & 14802 \\\hline
      Loads removed        & 20719 \\\hline
      Stores hoisted       & 15    \\\hline
      Stores removed       & 15    \\\hline
      Calls hoisted        & 8     \\\hline
      Calls removed        & 8     \\\hline
      Total Instructions hoisted & 21616 \\\hline
      Total Instructions removed & 30438 \\\hline
\end{tabular}
  \end{center}
  \caption{Code hoisting metrics on llvm-testsuite}
  \label{tab:hoist-results}
\end{table}

\begin{table}[h!]
  \begin{center}
    \begin{tabular}{|l|c|c|}
      \hline
      Metric               & Before & After              \\\hline
      Call sites deleted, not inlined             & 1988    & 1988   \\\hline
      Functions deleted (all callers found)       & 38250   & 38255  \\\hline
      Functions inlined                           & 154986  & 154985 \\\hline
      Allocas merged together                     & 212     & 212    \\\hline
      Caller-callers analyzed                     & 193042  & 193092 \\\hline
      Call sites analyzed                         & 414336  & 414381 \\\hline
      Rematerialized defs for spilling            & 18321   & 18326  \\\hline
      Rematerialized defs for splitting           & 5719    & 5842   \\\hline
      Spill slots allocated                       & 42912   & 42970  \\\hline
      Spilled live ranges                         & 61330   & 61362  \\\hline
      Spills inserted                             & 50724   & 50784  \\\hline
\end{tabular}
  \end{center}
  \caption{Static metrics before and after code-hoisting on llvm-testsuite}
  \label{tab:static-results}
\end{table}


Other static metrics are listed in Table~\ref{tab:static-results}. Here we can
see that except for rematerializing defs for splitting, which has an overhead of ~2\%, all
other parameters have less than 1\% overhead. This is to explain why the performance does not
go down with our implementation (and cost-model) of code hoisting pass.

\begin{table}[h!]
  \begin{center}
    \begin{tabular}{|l|c|}
      \hline
      Code-size metric  (.text)                   & Number   \\\hline
      Total  benchmarks                           & 497      \\\hline
      Total  gained in size                       & 39       \\\hline
      Total  decrease in size                     & 58       \\\hline
      Median decrease in size                     & 2.9\%    \\\hline
      Median increase in size                     & 2.4\%    \\\hline
    \end{tabular}
  \end{center}
  \caption{Code size metrics on llvm-testsuite}
  \label{tab:code-size}
\end{table}

While benchmarking llvm-testsuite we see both increase as well as decrease in
the codesizes of the final binaries. Since the pass runs early, it has affect on
many optimizations which rely on number of instructions, lenth of the use-def
chain etc metrics which are affected by code-hoisting. Specially the inliner
where the total inline cost (in llvm) is heavily dependent on the total count of
instructions in the caller and callee. Various code-size metrics are shown in
Table~\ref{tab:code-size}. All but one benchmark varied between -5.32\% and
5.43\%.  In one benchmark FreeBench/distray/distray.test, the codesize increased
by 35.38\%. In this benchmark 3 more (15 as compared to 12) functions got
inlined and because of that 10 more (81 vs. 71) vector instructions got
generated, 3 calls got hoisted/sunk as (compared to 0), one loop got unswitched
(compared to 0), 6 high latency machine instructions got hoisted out of loop, 59
(compared ot 30) machine instructions got hoisted out of loop, 70 (compared to
39) machine instructions were sunk.

\section{Conclusion and Future Work}
\label{sec:future-work}
We have presented the GVN based code hoisting algorithm. The primary goal is to
reduce the code size but it benefits performance in some cases as well. To
preserve performance and not hoist too much we have implemented several cost
models described in Section~\ref{sec:cost-models}. Since those cost models
depend on a set of thresholds, it requires tuning, as such, we used
representative benchmarks to tune them.

Currently, we rerun the algorithm until there are no more instructions left to
be hoisted. This is not the most optimal approach and results in expensive
analyses to be recomputed. This can be improved by ranking the computations
\cite{rosen1988global}. Also gvn-hoist runs very early in the pass pipeline, it
will be good to evaluate the codesize/performance impact when it is run in sync
with gvn-pre just like gcc does.

With the implementation of code-hoisting in llvm, the passes which rely on the
code-size/instruction-count to make optimization decisions needs to be
revisited. The first candidate would be the inliner. We have seen different
inlining decisions in Table~\ref{tab:code-size}, before and after
code-hoisting was enabled.  Since inliner has several magic numbers tuned for
the previous pass layout, it would need some improvement.

\section{Acknowledgments}
We would like to thank Daniel Berlin for his code reviews and for his feedback
on earlier versions of this paper.

\bibliographystyle{abbrv}
{\small
\bibliography{Bibliography}
}
\end{document}
