\documentclass{beamer}
\usepackage{hyperref}
\usepackage{graphicx}
\usepackage{amssymb}
\usepackage{graphviz}
\usepackage{tikz}
\usepackage{dot2texi}
\usepackage{listings}

\pgfdeclarelayer{background}
\pgfdeclarelayer{foreground}
\pgfsetlayers{background,main,foreground}

\begin{document}
\def \GVN {GVN}

\title{\GVN{}-Hoist: Hoisting Computations from Branches}
\author{Sebastian Pop and Aditya Kumar}
\institute{SARC: Samsung Austin R\&D Center}
\date{November 3, 2016}

\definecolor{myblue}{rgb}{0.0, 0.0, 0.5}
\definecolor{myred}{rgb}{0.5, 0.0, 0.0}
\definecolor{mygreen}{rgb}{0.0, 0.5, 0.0}
\lstset{language=C++,
  basicstyle=\ttfamily,
  keywordstyle=\color{myblue}\ttfamily,
  stringstyle=\color{myred}\ttfamily,
  commentstyle=\color{mygreen}\ttfamily,
  morecomment=[l][\color{magenta}]{\#}
}
\addtobeamertemplate{navigation symbols}{}{%
    \usebeamerfont{footline}%
    \usebeamercolor[fg]{footline}%
    \hspace{1em}%
    \insertframenumber/\inserttotalframenumber
}

\frame{\titlepage}



\frame{\frametitle{\GVN{}-Hoist: Hoisting Computations from Branches}
  \begin{itemize}
    \item Identifies identical computations in a function
    \item Hoist identical computations to a common dominator
    \item Uses GVN and MemorySSA
    \item Reduces code size
  \end{itemize}
}

\frame{\frametitle{\GVN{}-Hoist: Algorithm}
  \begin{itemize}
  \item dfs number all instructions
  \item compute value number of scalars, loads, stores, calls
  \item compute insertion points of each type of instructions separately
  \item hoist expressions
  \end{itemize}
}

\frame{\frametitle{\GVN{}-Hoist: Algorithm-collecting value numbers}
  \begin{itemize}
  \item For loads value number the pointer operand
  \item For stores value number the pointer operand and the value being stored
  \item For calls, scalars use the GVN infrastructure
  \end{itemize}
}

\frame{\frametitle{\GVN{}-Hoist: Algorithm-compute insertion points}
  \begin{itemize}
  \item For a set of instructions having the same value number, compute a common insertion point.
  \item When no common insertion points can be found, partition the candidates into a smaller set of
    hoistable candidates.
  \item Partition such that instructions with closeby DFS numbers are together.
  \end{itemize}
}

\frame{\frametitle{CFGSimplify's code hoisting}
  \begin{itemize}
  \item identifies computations to be hoisted at the beginning of BB
  \item stops at first difference
  \item very fast
  \item catches a lot of low hanging fruit
  \end{itemize}
}

\frame{\frametitle{Code size reduction}
  \begin{itemize}
  \item an average of 2.5\% savings in code size
  \item improve inline heuristics
  \item more functions inlined
  \end{itemize}
}

\end{document}
