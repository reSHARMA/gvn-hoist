\documentclass{beamer}
\usepackage{hyperref}
\usepackage{graphicx}
\usepackage{amssymb}
\usepackage{graphviz}
\usepackage{tikz}
\usepackage{dot2texi}
\usepackage{listings}
\usepackage{adjustbox}

\pgfdeclarelayer{background}
\pgfdeclarelayer{foreground}
\pgfsetlayers{background,main,foreground}

\begin{document}
\def \GVN {GVN}
\def \SSA {SSA}
\def \MemorySSA {Memory-\SSA{}}

\title{\GVN{}-Hoist: Hoisting Computations from Branches}
\author{Sebastian Pop and Aditya Kumar}
\institute{SARC: Samsung Austin R\&D Center}
\date{November 3, 2016}

\definecolor{myblue}{rgb}{0.0, 0.0, 0.5}
\definecolor{myred}{rgb}{0.5, 0.0, 0.0}
\definecolor{mygreen}{rgb}{0.0, 0.5, 0.0}
\lstset{language=C++,
  basicstyle=\ttfamily,
  keywordstyle=\color{myblue}\ttfamily,
  stringstyle=\color{myred}\ttfamily,
  commentstyle=\color{mygreen}\ttfamily,
  morecomment=[l][\color{magenta}]{\#}
}
\addtobeamertemplate{navigation symbols}{}{%
    \usebeamerfont{footline}%
    \usebeamercolor[fg]{footline}%
    \hspace{1em}%
    \insertframenumber/\inserttotalframenumber
}

\frame{\titlepage}

\frame{\frametitle{CFGSimplify's code hoisting}
  \begin{itemize}
  \item hoists computations at the beginning of BB
  \item uses operands equality to detect same computations
  \item stops at first difference
  \item very fast: disabling it slows the compiler: {\small $1688 \to 1692$ Bn insns
    (callgrind compiling the test-suite on x86\_64-linux)}
  \end{itemize}
}

\begin{frame}[fragile]{CFGSimplify limits}
  \begin{columns}[T,onlytextwidth] % align columns
    \column{.4\textwidth}
    \begin{block}{\small Original program}
      \begin{lstlisting}
inv = 1/d;
if (inv >= 0) {
  ta = a * inv;
  tb = b * inv;
} else {
  ta = b * inv;
  tb = a * inv;
}
      \end{lstlisting}
    \end{block}
    \pause
    \column{.1\textwidth}
    \begin{center}
      \vspace{2cm}
      $\longrightarrow$
    \end{center}
    \column{.4\textwidth}
    \begin{block}{\small Expressions hoisted}
      \begin{lstlisting}
inv = 1/d;
x = a * inv;
y = b * inv;
if (inv >= 0) {
  ta = x;
  tb = y;
} else {
  ta = y;
  tb = x;
}
      \end{lstlisting}
    \end{block}
  \end{columns}
\end{frame}

\frame{\frametitle{\GVN{}-Hoist: Hoisting Computations from Branches}
  \begin{itemize}
  \item removes all limitations of CFGSimplify implementation
  \item works across several BBs: hoists to a common dominator
  \item hoist past ld/st side effects: uses \MemorySSA{} for fast dependence analysis
  \item reduces code size
  \item reduces critical path length by exposing more ILP
  \end{itemize}
}

\frame{\frametitle{Optimistic \GVN{}-hoist Algorithm}
  \begin{enumerate}
  \item compute value number of scalars, loads, stores, calls
  \item compute insertion points of each type of instructions
  \item hoist expressions and propagate changes by updating \SSA{}
  \end{enumerate}
}

\begin{frame}[fragile]{\GVN{}: Value Numbering Example and Limitations}
  \begin{columns}[T,onlytextwidth] % align columns
    \column{.3\textwidth}
    \begin{block}{\small Simple program}
      \begin{lstlisting}
a = x + y
b = x + 1
c = y + 1
d = b + c
e = a + 2
f = load d
g = load e
      \end{lstlisting}
    \end{block}
    \pause
    \column{.1\textwidth}
    \begin{center}
      \vspace{1cm}
      $\longrightarrow$
    \end{center}
    \column{.4\textwidth}
    \begin{block}{\small Value Numbering}
      \begin{lstlisting}
(a, 1)
(b, 2)
(c, 3)
(d, 4)
(e, 4)
      \end{lstlisting}
    \end{block}
    \pause
    \begin{block}{\small Limitations to current \GVN{} implementation}
      \begin{lstlisting}
(f, 5)
(g, 6)
// should be (g, 5)
      \end{lstlisting}
    \end{block}
  \end{columns}

\end{frame}

\frame{\frametitle{\GVN{}-Hoist Step 1: Collect Value Numbers}
  \begin{itemize}
  \item scalars: use the existing \GVN{} infrastructure
  \end{itemize}
  current \GVN{} not accurate for loads and stores: use ad-hoc change
  \begin{itemize}
  \item loads: VN the gep
  \item stores: VN the gep and stored value
  \item calls: as stores, loads, or scalars (following calls' side-effects)
  \end{itemize}
}

\frame{\frametitle{\GVN{}-Hoist Step 2: Compute Insertion Points}
  insertion point: location where all the operands are available

  \begin{itemize}
  \item compute a common insertion point for a set of instructions having the same \GVN{}
    (similar to VBEs but not as strict)
  \item partition the candidates into a smaller set of hoistable candidates when
    no common insertion points can be found
  \end{itemize}
}

\frame{\frametitle{\GVN{}-Hoist Step 3: Move the Code}
  \begin{itemize}
  \item scalars: just move one of the instructions to the hoisting point and remove others; update \SSA{}
  \item loads and stores: make geps available, then hoist; update \SSA{} and \MemorySSA{}
  \end{itemize}
}

\newlength\someheight
\setlength\someheight{4cm}

\begin{frame}[fragile]{Example}
\begin{adjustbox}{width=\textwidth,height=\someheight,keepaspectratio}
\begin{lstlisting}
define float @f(float %d, float %min, float %max, float %a) {
entry:
  %div = fdiv float 1.000000e+00, %d
  %cmp = fcmp oge float %div, 0.000000e+00
  br i1 %cmp, label %if.then, label %if.else

if.then:                                          ; preds = %entry
  %sub = fsub float %min, %a
  %mul = fmul float %sub, %div
  %sub1 = fsub float %max, %a
  %mul2 = fmul float %sub1, %div
  br label %if.end

if.else:                                          ; preds = %entry
  %sub3 = fsub float %max, %a
  %mul4 = fmul float %sub3, %div
  %sub5 = fsub float %min, %a
  %mul6 = fmul float %sub5, %div
  br label %if.end

if.end:                                           ; preds = %if.else, %if.then
  %tmax.0 = phi float [ %mul2, %if.then ], [ %mul6, %if.else ]
  %tmin.0 = phi float [ %mul, %if.then ], [ %mul4, %if.else ]
  %add = fadd float %tmax.0, %tmin.0
  ret float %add
}
\end{lstlisting}
\end{adjustbox}
\end{frame}

\begin{frame}[fragile]{Example}
\begin{adjustbox}{width=\textwidth,height=\someheight,keepaspectratio}
\begin{lstlisting}
define float @f(float %d, float %min, float %max, float %a) {
entry:
  %div = fdiv float 1.000000e+00, %d
  %cmp = fcmp oge float %div, 0.000000e+00
  %sub1 = fsub float %max, %a
  %sub = fsub float %min, %a
  %mul2 = fmul float %sub1, %div
  %mul = fmul float %sub, %div
  br i1 %cmp, label %if.then, label %if.else

if.then:                                          ; preds = %entry
  br label %if.end

if.else:                                          ; preds = %entry
  br label %if.end

if.end:                                           ; preds = %if.else, %if.then
  %tmax.0 = phi float [ %mul2, %if.then ], [ %mul, %if.else ]
  %tmin.0 = phi float [ %mul, %if.then ], [ %mul2, %if.else ]
  %add = fadd float %tmax.0, %tmin.0
  ret float %add
}
\end{lstlisting}
\end{adjustbox}
\end{frame}


\frame{\frametitle{Cost models}
  tuned on x86\_64 and AArch64 Linux: test-suite, SPEC 2k, 2k6, \ldots{}
  \begin{itemize}
  \item limit the number of basic blocks in the path between initial position
    and the hoisting point
  \item limit the number of instructions between the initial position and the
    beginning of its basic block
  \item do not hoist GEPs (except at -Os)
  \item limit the number of dependent instructions to be hoisted
  \end{itemize}
}

\frame{\frametitle{Knobs}
  \begin{itemize}
  \item {\bf -enable-gvn-hoist:} enable the \GVN{}-hoist pass {\small(default = on)}
  \item {\bf -Os, -Oz:} allow GEPs to be hoisted independently of ld/st
  \item {\bf -gvn-hoist-max-bbs:} max number of basic blocks on the path between
    hoisting locations {\small (default = 4, unlimited = -1)}
  \item {\bf -gvn-hoist-max-depth:} hoist instructions from the beginning of the BB up
    to the maximum specified depth {\small (default = 100, unlimited = -1)}
  \item {\bf -gvn-hoist-max-chain-length:} maximum length of dependent chains to hoist
    {\small (default = 10, unlimited = -1)}
  \item {\bf -gvn-max-hoisted:} max number of instructions to hoist {\small (default unlimited = -1)}
  \end{itemize}
}
\frame{\frametitle{\GVN{}-Hoist: Evaluation}
  \begin{itemize}
  \item $<1\%$ compile time overhead: $1678 \to 1692$ Bn insns \\
    {\small (callgrind compiling the test-suite at -O3 on x86\_64-linux)}
  \item more hoists than CFG-simplify: $15048 \to 25318$ \\
    {\small (compiling the test-suite for x86\_64 at -O3)}
  \end{itemize}

  \begin{center}
    \begin{tabular}{|l|c|}
      \hline
      Scalars hoisted      & 8960  \\\hline
      Scalars removed      & 11940 \\\hline
      Loads hoisted        & 16301 \\\hline
      Loads removed        & 22690 \\\hline
      Stores hoisted       & 50    \\\hline
      Stores removed       & 50    \\\hline
      Calls hoisted        & 7     \\\hline
      Calls removed        & 7     \\\hline
      Total Instructions hoisted & 25318 \\\hline
      Total Instructions removed & 34687 \\\hline
    \end{tabular}
  \end{center}
}

\frame{\frametitle{Code size reduction}
  \begin{center}
    \begin{tabular}{|l|c|}
      \hline
      Code-size metric  (.text)                   & Number   \\\hline
      Total  benchmarks                           & 497      \\\hline
      Total  gained in size                       & 39       \\\hline
      Total  decrease in size                     & 58       \\\hline
      Median decrease in size                     & 2.9\%    \\\hline
      Median increase in size                     & 2.4\%    \\\hline
    \end{tabular}
  \end{center}
  \begin{itemize}
  \item test-suite compiled at -O3 for x86\_64-linux
  \item increase in size due to more inlining
  \item many effects due to early scheduling of the pass
  \end{itemize}
}

\frame{\frametitle{Discussion}
  \begin{itemize}
  \item schedule \GVN{}-hoist pass several times?
  \item remove CFGSimplify's hoisting?
  \item hoist + sink interactions (discuss with James Molloy)
  \item early scheduling in opt needs tuning with target info?
  \item make \GVN{}-hoist more aggressive for -Os and -Oz?
  \item need a better \GVN{} implementation?
  \item \MemorySSA{} is easy to use and fast: {\bf \color{myred}so please use it!} \\
    {\small(thanks Danny, Georges, and others)}
  \end{itemize}
}

\end{document}
