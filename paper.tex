%% For double-blind review submission
\documentclass[acmlarge,review,anonymous]{acmart}\settopmatter{printfolios=true}
%\documentclass[acmlarge,review,anonymous]{acmart}\settopmatter{printfolios=true}
%% For single-blind review submission
%\documentclass[acmlarge,review]{acmart}\settopmatter{printfolios=true}
%% For final camera-ready submission
%\documentclass[acmlarge]{acmart}\settopmatter{}

%% Note: Authors migrating a paper from PACMPL format to traditional
%% SIGPLAN proceedings format should change 'acmlarge' to
%% 'sigplan,10pt'.


%% Some recommended packages.
\usepackage{booktabs}   %% For formal tables:
                        %% http://ctan.org/pkg/booktabs
\usepackage{subcaption} %% For complex figures with subfigures/subcaptions
                        %% http://ctan.org/pkg/subcaption
\usepackage{comment}
\usepackage{amsmath}
\usepackage{hyperref}
\usepackage{graphicx}
\usepackage{amssymb}
\usepackage{graphviz}
\usepackage{auto-pst-pdf}
\usepackage{etoolbox}
\usepackage{flushend}
\usepackage{needspace}
\usepackage{algpseudocode}

\makeatletter\if@ACM@journal\makeatother

\makeatletter\preto{\@verbatim}{\topsep=1pt \partopsep=0pt}\makeatother

%% Journal information (used by PACMPL format)
%% Supplied to authors by publisher for camera-ready submission
\acmJournal{PACMPL}
\acmVolume{1}
\acmNumber{1}
\acmArticle{1}
\acmYear{2017}
\acmMonth{1}
\acmDOI{10.1145/nnnnnnn.nnnnnnn}
\startPage{1}
\else\makeatother
%% Conference information (used by SIGPLAN proceedings format)
%% Supplied to authors by publisher for camera-ready submission
\acmConference[PL'17]{ACM SIGPLAN Conference on Programming Languages}{January 01--03, 2017}{New York, NY, USA}
\acmYear{2017}
\acmISBN{978-x-xxxx-xxxx-x/YY/MM}
\acmDOI{10.1145/nnnnnnn.nnnnnnn}
\startPage{1}
\fi


%% Copyright information
%% Supplied to authors (based on authors' rights management selection;
%% see authors.acm.org) by publisher for camera-ready submission
\setcopyright{none}             %% For review submission
%\setcopyright{acmcopyright}
%\setcopyright{acmlicensed}
%\setcopyright{rightsretained}
%\copyrightyear{2017}           %% If different from \acmYear


%% Bibliography style
\bibliographystyle{ACM-Reference-Format}
%% Citation style
%% Note: author/year citations are required for papers published as an
%% issue of PACMPL.
\citestyle{acmauthoryear}   %% For author/year citations



\begin{document}


%% Title information
\title[SSA GCM]{Global code motion of congruent instructions}         %% [Short Title] is optional;
                                        %% when present, will be used in
                                        %% header instead of Full Title.
%\titlenote{with title note}             %% \titlenote is optional;
                                        %% can be repeated if necessary;
                                        %% contents suppressed with 'anonymous'
%\subtitle{Subtitle}                     %% \subtitle is optional
%\subtitlenote{with subtitle note}       %% \subtitlenote is optional;
                                        %% can be repeated if necessary;
                                        %% contents suppressed with 'anonymous'

\def \GCC {GCC}
\def \LLVM {LLVM}
\def \CFG {CFG}
\def \SSA {SSA}
\def \MemorySSA {MemorySSA}
\def \PRE {PRE}
\def \GVN {GVN}
\def \SPEC {SPEC Cpu 2006}
\def \gcm {global-code-motion}
\def \GCM {GCM}
\def \xlinux {x86\_64-linux}
\def \ooo {out-of-order}
%% Author information
%% Contents and number of authors suppressed with 'anonymous'.
%% Each author should be introduced by \author, followed by
%% \authornote (optional), \orcid (optional), \affiliation, and
%% \email.
%% An author may have multiple affiliations and/or emails; repeat the
%% appropriate command.
%% Many elements are not rendered, but should be provided for metadata
%% extraction tools.

%% Author with single affiliation.

\author{Aditya Kumar}
%\authornote{with author1 note}          %% \authornote is optional;
                                        %% can be repeated if necessary
%\orcid{nnnn-nnnn-nnnn-nnnn}             %% \orcid is optional
\affiliation{
  \position{Senior Compiler Engineer}
  %\department{Department1}              %% \department is recommended
  \institution{Samsung Austin R\&D Center}            %% \institution is required
  %\streetaddress{Street1 Address1}
  \city{Austin}
  \state{Texas}
  %\postcode{Post-Code1}
  \country{USA}
}
\email{aditya.k7@samsung.com}


\author{Sebastian Pop}
%\authornote{with author1 note}          %% \authornote is optional;
                                        %% can be repeated if necessary
%\orcid{nnnn-nnnn-nnnn-nnnn}             %% \orcid is optional
\affiliation{
  \position{Senior Staff Compiler Engineer}
  %\department{Department1}              %% \department is recommended
  \institution{Samsung Austin R\&D Center}            %% \institution is required
  %\streetaddress{Street1 Address1}
  \city{Austin}
  \state{Texas}
  %\postcode{Post-Code1}
  \country{USA}
}
\email{s.pop@samsung.com}

%% Paper note
%% The \thanks command may be used to create a "paper note" ---
%% similar to a title note or an author note, but not explicitly
%% associated with a particular element.  It will appear immediately
%% above the permission/copyright statement.
\thanks{with paper note}                %% \thanks is optional
                                        %% can be repeated if necesary
                                        %% contents suppressed with 'anonymous'


%% Abstract
%% Note: \begin{abstract}...\end{abstract} environment must come
%% before \maketitle command
\begin{abstract}
We present a \gcm{} (\GCM{}) compiler optimization which schedules
congruent instructions across the program.  Not only this saves code size, it
exposes redundancies in some cases, it exposes more instruction level
parallelism in the basic-block to which instructions are moved, and it enables
other passes like loop invariant motion to remove more redundancies. The cost
model to drive the code motion is based on live range analysis on \SSA{}
representation such that the (virtual) register pressure does not increase, which
translates to 2\% fewer spills while compiling \SPEC{} benchmark suite.

We have implemented the pass in \LLVM{}. It is based on Global Value Numbering
infrastructure available in \LLVM{}. The experimental results show an average
saving of 1\% on overall compile time on \SPEC{}. \GCM{} enables more inlining
and exposes more loop invariant code motion opportunities in majority of the
benchmarks. We have also seen execution time improvements in a few of \SPEC{}
benchmarks viz. mcf and sjeng, moreover, register spills reduced by 2\% on the
\SPEC{} benchmarks suite when compiled for \xlinux{}. This is an optimistic
algorithm in the sense that it considers all congruent instructions in a
function as potential candidates. We make an extra effort to hoist candidates by
partitioning the potential candidates in a way to enable partial hoisting in
case common hoisting points for all the candidates cannot be found. We also
formalize how register pressure will reduce as a result of \gcm{}, and how
\gcm{} increases instruction level parallelism thereby enabling more \ooo{}
execution on modern architectures.
\end{abstract}

%% 2012 ACM Computing Classification System (CSS) concepts
%% Generate at 'http://dl.acm.org/ccs/ccs.cfm'.
\begin{CCSXML}
<ccs2012>
<concept>
<concept_id>10011007.10011006.10011008</concept_id>
<concept_desc>Software and its engineering~General programming languages</concept_desc>
<concept_significance>500</concept_significance>
</concept>
<concept>
<concept_id>10003456.10003457.10003521.10003525</concept_id>
<concept_desc>Social and professional topics~History of programming languages</concept_desc>
<concept_significance>300</concept_significance>
</concept>
</ccs2012>
\end{CCSXML}

\ccsdesc[500]{Software and its engineering~Compilers}
\ccsdesc[500]{Software and its engineering~Source code generation}
\ccsdesc[500]{General and reference~Performance}

%\ccsdesc[300]{Social and professional topics~History of programming languages}
%% End of generated code


%% Keywords
%% comma separated list
\keywords{Optimizing Compilers, \GCC{}, \LLVM{}, Code Generation, Global Scheduling}  %% \keywords is optional

%% \maketitle
%% Note: \maketitle command must come after title commands, author
%% commands, abstract environment, Computing Classification System
%% environment and commands, and keywords command.
\maketitle


\section{Introduction}
\label{sec:intro}
Compiler techniques to remove redundant computations are composed of an analysis
phase that detects identical computations in the program and a transformation
phase that reduces the number of run-time computations.  Classical scalar
optimizations like CSE \cite{dragonbook} work very well on single basic-blocks but
when it comes to detect redundancies across basic-blocks these techniques fall
short: more complex passes like GCSE and \PRE{} have been designed to handle these
cases based on data-flow analysis \cite{morel1979global}.  At first these
techniques were described in the classical data-flow analysis framework, and
later the use of the \SSA{} representation lowered the cost in terms of compilation
time \cite{briggs1994effective,chow1997new,kennedy1999partial} and brought these
techniques in the main stream: nowadays \SSA{} based \PRE{} is available in every
industrial compiler.

This paper describes \gcm{} (\GCM{}) of congruent instructions \cite{briggs1997}, a
technique that uses the information computed for \PRE{} to detect identical
computations but has a transformation phase whose goal differs from \PRE{}: it
removes identical computations from different branches of execution.  These
identical computations in different branches of execution are not redundant
computations at run-time and the number of run-time computations is not
reduced. It is not a redundancy elimination pass, and thus it has different cost
function and heuristics than \PRE{} or CSE. It is more similar to global code
scheduling \cite{dragonbook,click1995global} in the sense that it will only move
computations. Code hoisting, for example, can reduce the critical path length of
execution in \ooo{} machines. As more instructions are available at the
hoisting point, the hardware has more independent instructions to reorder. The
following example illustrates how hoisting can improve performance by exposing
more ILP.

\begin{verbatim}
float fun(float d, float min, float max, float a) {
  float tmin, tmax;
  float inv = 1.0f / d;
  if (inv >= 0) { tmin = (min - a) * inv; tmax = (max - a) * inv; }
  else          { tmin = (max - a) * inv; tmax = (min - a) * inv; }
  return tmax + tmin;
}
\end{verbatim}

In this program the computations of tmax and tmin are identical to the
computations of tmin and tmax of sibling branch respectively. Both tmax and tmin
depend on inv which depends on a division operation which is generally more
expensive than the addition, subtraction, and multiplication operations. The
total latency of computation across each branch is: $C_{div} + 2(C_{sub} + C_{mul})$.
For an \ooo{} processor with two add units and two multiply
units, the total latency of computation is $C_{div} + C_{sub} + C_{mul}$.

Now if the computation of tmax and tmin are hoisted outside the
conditionals, the C code version would look like this:
\begin{verbatim}
float fun(float d, float min, float max, float a) {
  float inv = 1.0f / d;
  float x = (min - a) * inv;
  float y = (max - a) * inv;
  float tmin = x, tmax = y;
  if (inv < 0) { tmin = y; tmax = x; }
  return tmax + tmin;
}
\end{verbatim}

In this code the division operation can be executed in parallel with the two
subtractions because there are no dependencies among them. So the total number
of cycles will be $max(C_{div}, C_{sub}) + C_{mul} = C_{div} + C_{mul}$; since
$C_{div}$ is usually much greater than $C_{sub}$ \cite{x86,aarch64}. We have
seen the performance of an \ooo{} processor going up by $15\%$ on a benchmark
containing the above motivating example running in a tight loop.  In fact, there
are several advantages of \GCM{}:
\begin{itemize}
\item it helps reduce the code size of the program by replacing multiple
  instructions with one although the final code size may be higher because of
  improved inlining heuristics; functions become cheaper to inline when their
  code size decreases because the inliner heuristics depend on instruction
  count;
\item it exposes more instruction level parallelism to the later compiler
  passes. By hoisting identical computations to be executed earlier, instruction
  schedulers can move heavy computations earlier in order to avoid pipeline
  bubbles;
\item it reduces branch misprediction penalty on \ooo{} processors with
  speculative execution of branches: by hoisting or sinking expressions out of
  branches, it can effectively reduce the amount of code to be speculatively
  executed and hence reduce the critical path;
\item it reduces interference or register pressure with an appropriate
  cost-model: we have used a cost model in our implementation which only hoists
  or sinks when the register pressure is reduced;
\item it reduces the total number of instructions to be executed for SIMD
  architectures which execute all code in branches based on masking or
  predication;
\item it may improve loop vectorization by reducing a loop with control flow to
  a loop with a single basic-block, should all the instructions in a conditional get
  hoisted or sunk;
\item it enables more loop invariant code motion (LICM): as LICM passes, in
  general, cannot effectively reason about instructions within conditional
  branches in the context of loops, code-hoisting is needed to move instructions
  out of conditional expressions and expose them to LICM.
\end{itemize}

There has been a lot of work both in industry and academia to hoist and sink
code out of branches, and in general global scheduling
\cite{click1995global}. Some relate code-hoisting to code-size optimization
\cite{rosen1988global} and many \cite{barany2013, shobaki2013} use global
scheduling to improve performance. Most of the recent work on global scheduling
are done using integer linear programming (ILP), which results in prohibitively
high compile time. To the best of our knowledge we have not found any reference
which explored \gcm{} of congruent instructions in as much detail us. A part of
our implementation (aggressive code hoisting) is already merged
in \LLVM{} trunk, however, a general implementation of \gcm{} of congruent (non
redundant) expressions is still missing from \GCC{} and \LLVM{} trunk. The main
contributions of this paper are:
\begin{itemize}
\item a new optimistic algorithm to move congruent instructions out of branches;
\item a cost model to reduce interference and hence, reduce spills;
\item a technique to maximize hoisting in an optimistic approach by partitioning
  the list of potential candidates sorted by their DFS visit number;
\item experimental evaluation of our implementation in \LLVM{} which combines
  \SSA{} based liveness analysis, and ranking of expressions to move very busy
  expressions in order to reduce register spills and improve performance in some
  cases.
\end{itemize}

\section{Related Work}

There are a lot of bug reports in \GCC{} and \LLVM{} bugzillas
\cite{GCCCodeHoistingBugs, LLVMCodeHoistingBugs}, showing the interest in having
a more powerful code hoist transform.  The current \LLVM{} implementation of
code hoisting in SimplifyCFG.cpp is very limited to hoisting from identical
basic-blocks: the instructions of two sibling basic-blocks are read at the same
time, and all the instructions of the blocks are hoisted to the common parent
block as long as the compiler is able to prove that the instructions are
equivalent.  This implementation does not allow for an easy extension: first in
terms of compilation time overhead the implementation is quadratic in number of
instructions to bisimulate and second, the equivalence of instructions is
computed by comparing the operands which is neither general nor scalable.

\citet{dhamdhere1988fast} and \citet{steven1997advanced} mention
code hoisting in a data-flow framework. A list of Very Busy Expressions (VBE)
computed which are hoisted in a basic-block where the expression is
anticipable (all the operands are available). This algorithm would hoist as far
as possible without regarding the impact on register pressure and as such a cost
model will be required.  Also the description of VBE is based on the classic
data-flow model and an adaptation to a sparse \SSA{} representation is required.

\citet{rosen1988global} also briefly discuss hoisting computations with
identical value numbers from immediate successors. Their algorithm iterates on
computations of same rank and move the code with identical computations from the
sibling branch to a common dominator if they are very busy
\cite{steven1997advanced}. The cost-model to mitigate register pressure is
missing, also there is no mention of sinking congruent instructions.

\GCC{} recently got code-hoisting \cite{GCCCodeHoisting} which is implemented as
part of \GVN{}-\PRE{}: it uses the set of ANTIC\_IN and AVAIL\_OUT value
expressions computed for \PRE{}. ANTIC\_IN[B] contains very busy expressions at
basic-block B, i.e., values computed on all paths from B to exit and
AVAIL\_OUT[B] contains values which are already available. The algorithm hoists
top down to a predecessor.  It uses ANTIC\_IN[B] to know what expressions will
be computed on every path from B to exit, and can be computed in B.  It uses
AVAIL\_OUT[B] to subtract out those values already being computed.  The cost
function is: for each hoist candidate, if all successors of B are dominated by
B, then we know insertion into B will eliminate all the remaining computations.
It then checks to see if at least one successor of B has the value available.
This avoids hoisting it way up the chain to ANTIC\_IN[B].  It also checks to
ensure that B has multiple successors, since hoisting in a straight line is
pointless.  The algorithm continues top down the dominator tree, working in
tandem with \PRE{} until no more hoisting is possible.  One advantage of \GCC{}
implementation is that it works in sync with the \GVN{}-\PRE{} such that when
new hoisting opportunities are created by \GVN{}-\PRE{}, code-hoisting will
hoist them.

\citet{click1995global} describes aggressive \gcm{} to first
schedule all the instructions as early as possible. This results in very long
live ranges which is mitigated by again scheduling all the instructions as late
as possible. They report a speedup of as high as $23\%$.

\citet{barany2013} presented a global scheduler with ILP formulation with
a goal to minimize register pressure. The results they got were not very
promising. It may be because they only used the scheduler for smaller functions
($<1000$ instructions); also, they compiled the benchmarks for ARM-Cortex which
is more resilient to register pressure because it has more registers compared to
X86, for example.

\citet{shobaki2013} also recently presented a combinatorial global
scheduler with reasonable performance improvements. It is possible that both
Barany and Shobaki's implementation will have similar results when compiled for
same target architecture. Also, both suffer from the same problem, although
Shobaki not so much, of large compile times which would not suit in industrial
compilers like \GCC{} and \LLVM{}. With the algorithm described in the next
section, we got a reduction in register spills, improved inliner heuristics,
improved compile time, and show performance improvements on \SPEC{} benchmarks.

% TODO: References
% Sink instructions
% Ranking of instructions from Zadeck
% Liveness analysis from Ramakrishna
% Experimental results: Register spills, compile time, run time.

\section{Global code motion}

The algorithm for \gcm{} uses several common representations of the
program that we shortly describe below:
\begin{itemize}
\item Dominance (DOM) and Post-Dominance (PDOM) relations \cite{dragonbook} on a
  Control Flow Graph (\CFG{});
\item DJ-Graph \cite{Sreedhar1996} is a data structure that augments the
  dominator tree with join-edges to keep track of data-flow in a program. We use
  DJ-Graph to compute liveness of variables as illustrated in \cite{das2012};
\item Static Single Assignment (\SSA{}) \cite{cytron};
\item Global Value Numbering (\GVN{}) \cite{rosen1988global,click1995global}: to
  identify similar computations compilers use \GVN{}.  Each expression is given
  a unique number and the expressions that the compiler can prove to be
  identical are given the same number;
\item \MemorySSA{} \cite{novillo2007memory}: it is a factored use-def chain of
  memory operations that the compiler is unable to prove
  independent. \MemorySSA{} accelerates the access to the alias analysis
  information.
\end{itemize}

Dominance graphs, Global Value Numbers and Memory SSA are already available in
\LLVM{} but there is no facility to infer liveness of virtual registers. For
that we implemented DJ-Graph data structure that allows us to calculate
MergeSets which is used in the liveness analysis in our implementation of \GCM{}.

\subsection{Liveness using DJ-graph}
DJ-graph \cite{Sreedhar1996} is a data structure that augments the dominator
tree with join-edges to keep track of data-flow in a program. We use DJ-graph
and merge-sets to compute liveness of variables as illustrated in
\cite{das2012}.  It is very efficient for computing liveness and does not
require any bitvectors to be maintained for each basic-block. The underlying
simplicity of liveness computation is due to \SSA{} form, where the the values
only flow (from def to use) either through dominator edges or the join edges
(where we insert a PHI). The DJ-graph contains both these edges which allows for
computation of merge-sets for each basic-block, i.e., a set of all basic-blocks
where the values can flow from a particular basic-block. We have implemented the
merge-set computation based on a DJ-graph as illustrated in \cite{das2005}. A
simplified version of the merge-set that we implemented is presented here:

\begin{verbatim}
// Return true if the merge-set of source node of a visited J-edge (incoming edge of
// lnode) is not the subset of the merge-set of lnode.
bool still_inconsistent(Node lnode, JEdges JE, Visited V, MergeSet MergeSet)
  for e in (all incoming edges to lnode)
    if (e is in JE && e in V)
      if (MergeSet(Source Node of e) is not a subset of MergeSet(lnode))
        return true
  return false

// Compute a merge-set top-down in breadth first order.
bool constructMergeSet_1(BFSList B, JEdges JE, DomLevel DL)
  repeat = false
  V = empty list of visited edges
  for n in B
    for e in (all incoming edges to n)
      if (e is in JE && e not in V)
        V(e) = true
        let tmp = Source Node of e
        let tnode = Target Node of e
        let lnode = NULL
        while (DL(tmp) >= DL(tnode))
          MergeSet(tmp) = MergeSet(tmp) U MergeSet(tnode) U {tnode}
          lnode = tmp
          tmp = dom-parent(tmp)
        repeat = still_inconsistent(lnode, JE, MergeSet)
  return repeat

// Construct merge-set of each node in Control-Flow-Graph G of a function.
void constructMergeSet(CFG G)
  B = Breadth First Order of G
  JE = JEdges of G
  DL = List of Path length (from root) of each node in G.
  do // Call until a fixed point is reached.
    Repeat = constructMergeSet_1(B, JE, DL)
  while (Repeat)
\end{verbatim}

We compute merge sets, of the control flow graph, which remain same throughout
the \gcm{} transformation. For \gcm{} we only want to know if a use
operand is a kill (to compute changes in register pressure.) For that we only
need to know whether the use is also required later in the execution path. A
variable is live out of a basic-block B if it is used in the merge set of B. We
compute the live-out relation on-demand when profitability of hoist/sink
candidates is to be evaluated. A simplified version of
\emph{isLiveOutUsingMergeSet} is presented here:

\begin{verbatim}
// Return true if variable A is liveout from basic-block N
bool isLiveOutUsingMergeSet(Node N, Variable A)
  if (A is defined in N)
    if (A is used outside any basic-block other than N)
      return true
    return false

  MergeSet(N) = null
  // The merge-set of N is the union of merge-sets of its successors.
  for W in successors(N)
    MergeSet(N) = MergeSet(N) U MergeSet(W)

  // Iterate over all the uses of A and see if any intersect with the merge-set of N.
  for T in users(A)
    B = basic_block(T)
    while (B != null and B != def_bb(A))
      if B in MergeSet(N) return true
      B = dom-parent(B)
  return false

// Return true if op is the last use of the operand in I.
bool isKill(Operand op, Instruction I)
  if (isLiveOutUsingMergeSet(op))
    return false

  for each I1 after I in basic_block(I)
    if (I1 uses op)
      return false
  return true
\end{verbatim}

The original algorithm presented in \cite{das2012} has mismatched types in terms
of uses and nodes because each node can have many instructions and hence many
uses. Also, while iterating on the dominator of each user of a variable we may
reach to the beginning of a function, in that case the inner while loop needs to
terminate. These two cases were missing from the algorithm and we came across
them during implementation. If a variable is not live-out of a basic-block, it
still may be used later in the basic-block: to establish whether a use of a
variable is a kill we iterate on all the subsequent instructions in a
basic-block checking for uses, the \emph{isKill} function uses
\emph{isLiveOutUsingMergeSet} to find out if an operand is killed in an
instruction.

The \GCM{} pass can be broadly divided into the following steps that we will
describe in the rest of this section:
\begin{itemize}
\item find candidates (congruent instructions) suitable for \gcm{} (Section~\ref{subsec:finding-candidates}),
\item compute a point in the program where it is both legal
  (Section~\ref{subsec:legality}) and profitable
  (Section~\ref{subsec:cost-models}) to move the code,
\item move the code to hoist point or sink point
  (Section~\ref{subsec:optimistic}), and
\item update data structures to continue iterative \gcm{} (Section~\ref{subsec:optimistic}).
\end{itemize}

\subsection{Finding candidates to move}
\label{subsec:finding-candidates}
The first step is to find a set of congruent instructions
\cite{briggs1997}. This is performed by a linear scan of all instructions of the
program and classifying them by their value numbers. We could compute available
and anticipable sets as computed by \GCC{'}s code-hoisting but that would be a lot
of data structures to maintain at each basic-block level. For \GCC{} it makes sense
because their code-hoisting is integrated with \GVN{}-\PRE{} which already has those
data structures available.

The current implementation of \GVN{} in \LLVM{} has some limitations when it
comes to loads and stores so we compute the \GVN{} of loads and stores
separately.  Our solution is to value number the address from where the value of
a load is to be read from memory. For stores, we value number the address as
well as the value to be stored at that address. Another limitation of the
current \GVN{} implementation in \LLVM{} is that the instructions dependent on
the loads will not get numbered correctly, and so after hoisting all candidates
we need to rerun the \GVN{} analysis in order to discover new candidates now
available after having hoisted load instructions.  This limitation should be
addressed in a new implementation of the \GVN{} based on \MemorySSA{}, that
would better account for equivalent loads and their dependent instructions.

The process of computing \GVN{} can be on-demand (as we come across an
instruction) or, precomputed (computing \GVN{} of all the instructions
beforehand). Which process to choose is determined by the scope of code-hoisting
we want to perform. In a pessimistic approach, we want to hoist a limited set of
instructions from the sibling branches as we iterate the DFS tree bottom-up, it
is sufficient to compute \GVN{} values on-demand. Whereas, in the optimistic
approach, as described in Section~\ref{subsec:optimistic}, we want to move as
many instructions as possible, and it would require \GVN{} values to be
precomputed.

Once the instructions have been classified into congruence classes, we compute
for each group of congruent instructions, a point in the program that is both
legal and profitable for the instructions to be moved to.

\subsection{Legality check}
\label{subsec:legality}
Since the equality of candidates is purely based on the value numbers, we also
need to establish if hoisting them to a common dominator or sinking them to a
common post-dominator would be legal. Once a common dominator (post-dominator)
is found, we check whether all the use-operands of the set of instructions are
available at that position. It is possible to reinstantiate or remateralize the
use-operands in some cases when the operands are not available and make it legal
to move the instruction.

Subsequently, it is checked that the side-effects of the computations (if any)
do not intersect with any side-effects between the instructions to be
hoisted/sunk and their hoisting/sinking point. It is also necessary to check if
there are indirect branch targets, e.g., landing pad, case statements, goto
labels etc., along the path because it becomes difficult to prove safety checks
in those cases. In our current implementation we discard candidates on those
paths.

\subsubsection{Legality of hoisting scalars}
Scalars are the easiest to hoist because we do not have to analyze them for
aliasing memory references. As long as all the operands are available (or can be
made available by rematerialization), the scalar computations can be hoisted.

\subsubsection{Legality of hoisting loads}
The availability of operand to the load (an address) is checked at the hoisting
point. If that is not available we try to rematerialize the address if
possible.  Along the path, from current position of the load instruction
backwards on the control flow to the hoisting point, we check whether there are
writes to memory that may alias with the load, in which case the candidate is
discarded. To iterate on the use-def chains for memory references the MemorySSA
infrastructure of \LLVM{} is used.

\subsubsection{Legality of hoisting stores}
For stores, we check the dependency requirements similar to the hoisting of
loads using the \MemorySSA{} of \LLVM{}. We check that the operands of the store
instruction are available at the hoisting point, and that there are no aliasing
loads or store along the path from the current position to the hoisting point.

\subsubsection{Legality of hoisting calls}
Call instructions can be divided into three categories: those calls equivalent
to purely scalar computations, calls reading from memory, and most of the time,
without further information, calls have to be classified as writing to memory,
that is the most restrictive form.  Each category of call instructions is
handled as described for scalar, load, and store instructions.

Hoisting loads/stores across calls also require precise analysis of all the
memory addresses accessed by the call. The current implementation being an
intraprocedural pass, cannot hoist aggressively across calls. In the presence of
pure calls, loads can be hoisted but stores can't. Also, if a call throws
exceptions, or if it may not return, nothing can be hoisted across that call.

\subsubsection{Legality of sinking expressions}
\label{subsec:legality-sink}
Sometimes, hoisting is not upward-safe \cite{click1995global}, e.g., if the
expressions are in a landing pad, sinking of those expressions
may reduce code size. Sinking may also reduce the register pressure in some
cases, e.g., when the use operands are not kills. For sinking, higher ranked
expressions would be sunk first. And it would be illegal to sink higher ranked
identical expressions if they are not anticipable in the common
post-dominator. For example:

\begin{verbatim}
B0: i0 = load B
B1: i1 = load A
    c1 = i1 + 10
    d1 = i0 + 20
    goto B3
B2: i2 = load A
    c2 = i2 + 10
    d2 = i0 + 20
    goto B3

B3: phi(c1, c2)
    phi(d1, d2)
\end{verbatim}

In this example (c1, c2) or (d1, d2) are potential sinkable candidates. Since
(c1, c2) depend on i1 and i2 respectively which are also in their original basic
blocks, c1 and c2 are not anticipable in B3. So without knowing the ability to sink
of `i1' and `i2' it would be illegal to sink (c1, c2) to B3. On the other hand
(d1, d2) can safely be sunk because their operands are readily available at the
sink point, i.e., B3. It should also be noted that, just because the expressions
are identical and operands are available, it still requires a unique
post-dominating PHI to use the exact same values to be legally sinkable.

A general global scheduling algorithm also requires checks for undefined values
when introducing a new computation along a path. Since only very busy
expressions are moved in the current implementation, there is no need to check
for undefined resulting due to movement of instructions. This simplifies the
implementation.

\subsubsection{Barriers}
Barriers are based on the concept of pinned instructions \cite{click1995global}
but extended to adapt to \LLVM{} IR. Since a basic-block in \LLVM{} IR is
actually an extended basic-block because there might be calls in the middle of a
basic-block which might not return. Essentially, any instruction that cannot
guarantee progress is marked as a barrier. In the absence of context, as in our
current implementation, some instructions which might be safe are still
classified as barriers, e.g., volatile loads/stores, calls with missing
attributes.

\begin{verbatim}
computing barriers:
for each basic-block B in a function:
  barrier_found = false
  for each instruction I in B:
    if I does not guarantee progress:
      mark I as a barrier instruction
      barrier_found = true
      break;

  // Find the last barrier below which instructions can be sunk.
  // If there was no barrier in B, any instruction satisfying
  // other legality checks can be sunk.
  if barrier_found:
    for each instruction I in B in reverse order:
      if I does not guarantee progress:
        mark I as a sink barrier
        break
\end{verbatim}

\subsubsection{Downward-safety of movable instructions}

In order to be able to move an instruction that has side effects, we must
guarantee that the side-effects of that instructions before and after its motion
will produce the same side-effects.  For hoisting in particular, we need to
prove that on all execution paths from the hoisting point to the end of the
program the side effects appear in the same order.

\begin{verbatim}
computing downward-safety of hoistable instructions at hoist-point
  Worklist = list of all basic-blocks of hoistable instructions
  for each basic-block B in the dominator tree starting at hoist-point:
    if Worklist is empty, return false // Path exists! not downward safe
    if B is in Worklist:
      remove B from worklist
      remove subtree with root at B // Available at B => downward safe from B
    if B is a leaf basic-block, return false  // Path exists! not downward safe
    if B dominates hoist-point, return false  // Back edge
  return true
\end{verbatim}

\subsection{Profitability check (Cost models)}
\label{subsec:cost-models}
After the legality checks have passed, we check if a \gcm{} is profitable.
That takes into account the impact \gcm{} would have on various parameters
that might affect runtime performance, e.g., impact on live-range, gain in the
code size. Since this is mostly a code-size optimization pass, the goal is to
not regress in performance across popular benchmarks and at the same time to reduce
code size as much as possible. Following are the cost models which are
implemented:

\subsubsection{Reduce register pressure}
\label{hoist:reg-pressure}
Hoisting upwards will decrease the live-range of its use, if it is a last use (a
kill), but increase the live-range of its definition. Conversely, sinking will
decrease the live-range of the defined register but increase the live-range for
killed operands. If the live-range after \gcm{} is less than before it will
be moved. Essentially, as long as there is one killed operand, code hoisting
will either decrease or preserve the register pressure.  Similarly, code-sinking
will either decrease or preserve the register pressure as long as there is one
operand killed at most.  The following example explains how \gcm{} of
identical computations can reduce the register pressure.

\begin{verbatim}
B0: b = m
    c = n
    if c is true then goto B3 else goto B4

B1: a0 = b<kill> + c<kill>

B2: a1 = b<kill> + c<kill>
\end{verbatim}

After hoisting a0 and a1 are removed and a copy of a0 as a01 is placed in B0
just before the branch. Since `b' and `c' are killed in `a0' and `a1', hoisting
the expressions will reduce the register pressure because two registers will be
freed.

Ideally, it should be okay to hoist all the instructions and a later
live-range-splitting \cite{cooper1998live} pass should make the right decision
of rematerializing the instruction should it be beneficial to do so. But the
current live-range splitting pass of \LLVM{} is not making the optimal decision
and we have found performance regressions while hoisting aggressively.

Moreover, \LLVM{} has a `getelementptr' instruction which computes the address
where a load or a store would happen. It is a scalar computation and gets
hoisted frequently even if the corresponding loads/stores would not be legal to
hoist.  In order to reduce the register pressure while hoisting loads, we have
restricted hoisting of address computations away from their corresponding loads
and stores when the loads and stores cannot be moved. This restriction is only
to mitigate the limitations of \LLVM{'}s register allocator and may be lifted in
the future, when the register allocation rematerialization pass has been
improved to catch these regressions.

\subsubsection{Hoisting an expression across a call}
\label{cost:across-calls}
Even hoisting scalars across calls is tricky because it can increase the number
of spills. During the frame lowering of calls, the argument registers also known
as the caller saved registers are saved because they might be modified by the
callee and after the call they are restored. So before the call, the register
pressure is high because the number of available registers are reduced by the
number of caller saved registers. In that situation a computation that increases
register pressure is not profitable to hoist.

\subsubsection{Partitioning the list of hoist candidates to maximize hoisting}
\label{subsec:partition}
%The partition algorithm is described as follows:
In the approach, described in Section~\ref{subsec:optimistic}, it is possible
that a common hoisting point for all the instructions does not satisfy legality
or profitability checks. In these cases, it is still possible to `partially'
hoist a subset of instructions by splitting the set of candidates and finding a
closer hoisting point for each subset.

In order to hoist a subset of identical instructions, we partition the list of
all candidates in a way to maximize the total number of hoist instructions.  By sorting
the list of all the candidates in the increasing order of their depth first
search discovery time stamp \cite{clrs} (DFSIn numbers), we make sure that
candidates closer in the list have their common dominator nearby. In
Figure-\ref{fig:dfsin}, if B3, B4, B5, and B6 have identical computations and if
for some reason, they cannot be hoisted at B0, then we partition the set of
hoistable candidates in their DFSIn order. In this case the DFSIn ordering would
be B3, B4, B5, B6 which will allow the instructions in (B3, B4) to be hoisted at
B1 and those in (B5, B6) to be hoisted at B2. Even if instructions in (B3, B4)
could not be hoisted, for some reason, to B1, the instructions in (B5, B6) can
still be hoisted -- if legal and profitable -- to B2.

\begin{figure}
\centering
%rankdir=LR;
\digraph[scale=0.35]{abc}{
  B0 -> B1
  B0 -> B2
  B1 -> B3
  B1 -> B4
  B2 -> B5
  B2 -> B6
}
\caption{\CFG{} to illustrate partitioning}
\label{fig:dfsin}
\end{figure}

% TODO: Rewrite the following.
In our current implementation we keep as many candidates in one set as possible
(greedy approach.) We split the list at a point where the legality checks fail
to hoist subset of candidates which are legal to hoist and then start finding
new hoisting point for the remaining ones.


There is a special case for hoisting load instructions when the hoist-point is
the predecessor basic-block for all the loads. Even if the register pressure
would increase, we prefer to hoist loads in this case. The reason being, that
would make the loaded value available by the time it is used. Also, because
stores and calls are hoisted the least, see Table~\ref{tab:code-motion-metric},
the performance does not change much whether they are hoisted or not.

\subsection{Code generation for the optimistic \gcm{} algorithm}
\label{subsec:optimistic}
Once all the legality and profitability checks are satisfied for a set of
congruent instructions, they are suitable candidates for hoisting or sinking. A
copy of the computation is inserted at the hoisting/sinking point along with any
instructions which needed to be rematerialized. Thereafter, all the computations
made redundant by the new copy are removed, and the \SSA{} form is restored by
updating the intermediate representation (IR) to reflect the changes. At the
same time \MemorySSA{} is also updated to get up-to-date information about
memory references.

After one iteration of algorithm runs through the entire function, it creates
more opportunities for \emph{higher ranked} computations
\cite{rosen1988global}. Currently, this is a limitation of the \GVN{} analysis
pass, and so we rerun the code-hoisting algorithm until there are no more
instructions left to be hoisted.  Obviously, this is not the most optimal
approach and can be improved by ranking the computations \cite{rosen1988global},
or by improving the \GVN{} analysis to correctly populate congruence classes as
the program is modified by the code generation.

Finally after the transformation is done, we verify a set of post-conditions to
establish that program invariants are maintained: e.g., consistency of use-defs,
and \SSA{} semantics.

The amount of hoisting depends on whether we collect \GVN{} of instructions
before finding candidates (optimistic) or, on-demand (pessimistic). It also
depends on the generality of the \GVN{} algorithm as mentioned earlier in
Section~\ref{subsec:finding-candidates}. We have implemented an optimistic
\gcm{} of congruent instructions which uses the liveness analysis as
illustrated by \citet{das2012} and ranking expressions explained by
\citet{rosen1988global}.

\GCM{} basically consists of two parts, i.e., hoisting and sinking. This
implementation only moves congruent instructions. An immediate guarantee of this
approach is gain in code size (the final executable may be of larger size
because of more inlining). The algorithm prefers hoisting to sinking. If the
dependency of a hoistable candidate is in the same basic-block as the candidate,
then the dependency must also be hoistable, otherwise hoisting will be illegal or
would require a complicated code generation to make it legal. The current
algorithm discards cases if the dependency is neither hoistable nor
rematerializable.

\begin{verbatim}
void doGCM()
  Analyses available: Dominator Tree, DFS Numbering, Memory SSA,
  Compute DJ-graph of function and compute mergeset based on that
  do
    Compute GVN of each expression in the function
  while (doCodeHoisting() > 0) // If any of candidates were hoisted then repeat hoisting
  doCodeSinking()
\end{verbatim}

We collect the \GVN{} of all the instructions in the function and iterate on the
list of instructions having identical \GVN{}s. The algorithm prefers hoisting to
sinking. So first we find the common dominator dominating all such identical
computations and perform legality checks, as described in
Section~\ref{subsec:legality}. Often times it is not possible to hoist all the
instructions to one common dominator, due to legality constraints, e.g.,
intersecting side-effects, or profitability constraints, e.g., increasing
register pressure; in those cases, this algorithm would partition
(Section~\ref{subsec:partition}) the list of identical instructions into subsets
which can be partially hoisted to their respective common dominators. However,
it should be noted that we only hoist very busy expressions to avoid checking
for undefined behaviors resulting because of introducing extra computation in an
execution path.

\begin{verbatim}
int doCodeHoisting()
  For each value number VN with 2 or more instructions (I1, I2, ..., Ik)
    sort the instructions according to DFSIn numbers
    if first two I1, I2 are hoistable and if they are downward-safe at hoist-point
       and if they are profitable to be hoisted at hoist-point
      proceed to check ability to hoist of subsequent instruction with same VN
    else
      proceed to check if I2 and I3 (if available) can be hoisted
  For each set of instructions (I1, ..., Ik) with same VN that can be hoisted:
    move the first one I1 to the hoist-point
    update the use of all other candidates (I2, ..., Ik) to refer to I1
    remove all other instructions (I2, ..., Ik)
    if I1 has memory references (a load, store, call, etc.)
      update MemorySSA of I1 and others to point to the MemorySSA reference of I1
      remove MemorySSA reference of all others which were deleted
    update statistics
  return number of hoisted instructions
\end{verbatim}

\newpage
After no more candidates are hoistable, sinking is performed. Sinking is only
performed once because currently there are very few sinkable candidates
(Table~\ref{tab:code-motion-metric}) as SimplifyCFG of LLVM (which runs before
\GCM{}) has a code-sinking which already sinks several instructions.

\begin{verbatim}
void doCodeSinking()
  For each value number VN with 2 or more instructions
    if there are two instructions with same immediate successor as post-dominator
      if both are only used in the same PHI-node (PN) of the successor
        if both have dependencies that are available or can be made available at the sink point
           mark VN as sinkable
  For each pair (I1, I2) of sinkable instructions
    Move I1 to the sink point which is just after all the PHI-nodes in the post-dominator
    update all the uses of PHI node (PN) with I1
    Remove I2 and PN
    if I1 is a memory reference:
      update MemorySSA of I1 and others to point to the MemorySSA reference of I1
      remove MemorySSA reference of all others which were deleted
    update statistics
\end{verbatim}

Code hoisting opens new opportunities for other hoistable candidates which were
of higher rank (depended on candidates which got hoisted). Ideally we could
iterate on lower ranking expressions first and then proceed to higher ranking
expressions in the same iteration but \LLVM{'}s \GVN{} infrastructure does not
compute equivalence classes in an effective way. We found it simpler to just
recompute the value numbers and start finding hoistable candidates again.

\subsection{Time complexity of algorithm}
The complexity of code hoisting is linear in number of instructions that are
candidates for \gcm{}, matching the complexity of \PRE{} on \SSA{} form.
The analyses computed for this pass are Global Value Numbering, Computation of
DJ-graph and merge-sets, Marking Barriers, all are linear in number of
instructions in a function. Liveness analysis is expensive and only performed
on-demand for hoistable candidates so it does not affect the overall compilation
time by much. Other analyses like Alias Analysis, \MemorySSA{} and Dominator Tree
are already available in the \LLVM{} pass pipeline. Although we recompute \GVN{}, and
Barriers for each iteration of the \gcm{}, we have not seen significant
increase in the compilation times (see Section~\ref{tab:compile-time}). We have also
provided appropriate compiler
flags to expedite \gcm{} by bailing out with fewer iterations, or skip the
liveness based profitability analysis to aggressively move the code as long as
they are legal.

The analysis is followed by a simple code generation that adds the identified
instruction in the destination point and removes all instructions rendered
redundant as a result of \gcm{}.


\section{Experimental Evaluation}
\label{sec:experimental-results}
% Experimental results: Register spills, compile time, run time. compile-time comparison w.r.t. gcc
For evaluation of \gcm{}, we built \SPEC{} with the patch. All the experiments
were conducted on an \xlinux{} machine and at -Ofast optimization level. Each
sub benchmark was run three times and the best result was taken. We collected
execution time and code-size which is listed in Table~\ref{tab:code-size}, other
compiler stats are listed in Table~\ref{tab:spills}, Table~\ref{tab:stats}, and
Table~\ref{tab:code-motion-metric}.  We also measured compile time
Table~\ref{tab:compile-time} to see the impact of \gcm{}.

\begin{table}[h!]
  \begin{center}
    \begin{tabular}{|l|c|c|c|}
      \hline
% TODO: Check if putting code-size or spills number is okay or only %age is required.
% llvm trunk vs. llvm trunk with code-motion
                &\% performance uplift & \% code size decrease & \% code size decrease\\
Benchmarks      & at -Ofast (high better)& at -Ofast (low better) & at -Os (low better)\\\hline
400.perlbench	& 1.00  & 1.00 & 1.00\\\hline
401.bzip2	& 1.00  & 1.00 & 1.00\\\hline
403.gcc	        & 1.00  & 1.01 & 1.00\\\hline
429.mcf	        & 1.03  & 1.00 & 1.00\\\hline
433.milc	& 1.00  & 1.00 & 1.00\\\hline
444.namd	& 1.00  & 1.01 & 1.00\\\hline
445.gobmk	& 0.99  & 1.01 & 1.00\\\hline
447.dealII	& 0.95  & 1.01 & 0.98\\\hline
450.soplex	& 0.99  & 0.99 & 1.00\\\hline
453.povray	& 0.97  & 1.00 & 1.00\\\hline
456.hmmer	& 0.99  & 1.00 & 1.00\\\hline
458.sjeng	& 1.02  & 1.00 & 1.00\\\hline
462.libquantum	& 1.00  & 1.00 & 1.00\\\hline
464.h264ref	& 1.00  & 1.00 & 1.00\\\hline
470.lbm	        & 1.00  & 1.00 & 1.00\\\hline
471.omnetpp	& 1.00  & 0.99 & 1.00\\\hline
473.astar	& 1.00  & 1.02 & 1.00\\\hline
482.sphinx3	& 0.99  & 1.00 & 1.00\\\hline
483.xalancbmk	& 1.01  & 0.99 & 1.00\\\hline
Geomean         & 0.997 & 1.00 & 1.00\\\hline
    \end{tabular}
  \end{center}
  \caption{Execution time and code size with and without \gcm{}
    (\GCM{}) on \SPEC{}: performance uplift at -Ofast in mcf and sjeng; code size
    improvement in dealII at -Os.}
  \label{tab:code-size}
\end{table}

Overall there was only a minor change in the performance numbers with a few improvements 
(429.mcf, 458.sjeng etc.) and a few regressions (447.dealII, 453.povray
etc.). The \gcm{} pass was run twice in the pass pipeline (at -Ofast),
first after EarlyCSE pass, and then after the inliner. Because EarlyCSE removes
local redundancies, \GCM{} would have to analyze less redundant instructions, also
as inliner creates more redundancies, \GCM{} would have more opportunities to move
congruent instructions out of branches.

Code size also remained almost the same as a result of \gcm{} as shown in
Table~\ref{tab:code-size}.  The code size listed here is the text size (from
linux size command) of the final executable for each benchmark.  Some benchmarks
like 473.astar increased in code size by almost 2\%. This is because once the
code-size of a function decreases (due to \GCM{}), it becomes cheaper to
inline. As we can see in Table~\ref{tab:stats}, four more functions got inlined
in 473.astar. Except for 447.dealII, 450.soplex, 482.sphinx3 and 483.xalancbmk
all other benchmarks got more functions inlined. Also, since the pass runs
early, it affects many optimizations which rely on the number of instructions,
length of the use-def chain, and other metrics.

\begin{table}[h!]
  \begin{center}
    \begin{tabular}{|l|r|r|r|}
      \hline
Number of spills  &base(-Ofast) &\GCM{}(-Ofast) &\%change (low better) \\\hline
400.perlbench	  & 2539	& 2493	& 0.98 \\\hline
401.bzip2	  & 718	        & 707	& 0.98 \\\hline
403.gcc	          & 5782	& 5722	& 0.99 \\\hline
429.mcf	          & 14	        & 17	& 1.21 \\\hline
433.milc	  & 635	        & 642	& 1.01 \\\hline
444.namd	  & 3223	& 3274	& 1.01 \\\hline
445.gobmk	  & 2166	& 2173	& 1.00 \\\hline
447.dealII	  & 11074	& 10234	& 0.92 \\\hline
450.soplex	  & 1122	& 1125	& 1.00 \\\hline
453.povray	  & 4701	& 4692	& 1.09 \\\hline
456.hmmer	  & 1197	& 1274	& 1.06 \\\hline
458.sjeng	  & 168	        & 176	& 1.05 \\\hline
462.libquantum	  & 118	        & 118	& 1.00 \\\hline
464.h264ref	  & 3379	& 3454	& 1.02 \\\hline
470.lbm	          & 33	        & 33	& 1.00 \\\hline
471.omnetpp	  & 524	        & 527	& 1.00 \\\hline
473.astar	  & 180	        & 201	& 1.12 \\\hline
482.sphinx3	  & 674	        & 670	& 0.99 \\\hline
483.xalancbmk	  & 5102	& 4965	& 0.97 \\\hline
Grand Total	  & 43349	& 42497	& 0.98 \\\hline
    \end{tabular}
  \end{center}
  \caption{Number of spills before and after \gcm{}(\GCM{}) on SPEC2006 at
    -Ofast. Spills reduced by 2\%.}
  \label{tab:spills}
\end{table}

Spills are listed in Table~\ref{tab:spills}. Spills were reduced by an average
of 2\% on the \SPEC{} benchmark suite. This is a result of cost-model which
limits the \gcm{} to those candidates which do not increase the register
pressure (except for loads Section~\ref{subsec:cost-models},) overall spills
still decreases.

\begin{table}[h]
  \begin{center}
    \begin{tabular}{|l|c|c|c|}
      \hline
  Benchmarks      &   Baseline	        & \GCM{}           & \%Decrease (low better) \\\hline
  400.perlbench	  &   186,829,426	& 184,690,006	& 0.99 \\\hline
  401.bzip2	  &   76,337,033	& 75,942,641	& 0.99 \\\hline
  403.gcc         &   1,079,625,231	& 1,073,158,220	& 0.99 \\\hline
  429.mcf         &   78,336,298	& 77,859,513	& 0.99 \\\hline
  433.milc	  &   272,525,306	& 269,684,876	& 0.99 \\\hline
  444.namd	  &   78,354,313	& 77,875,603	& 0.99 \\\hline
  445.gobmk	  &   261,229,078	& 258,459,798	& 0.99 \\\hline
  447.dealII	  &   643,457,797	& 638,628,331	& 0.99 \\\hline
  450.soplex	  &   245,442,755	& 242,811,500	& 0.99 \\\hline
  453.povray	  &   499,034,715	& 494,785,035	& 0.99 \\\hline
  456.hmmer	  &   214,425,324	& 211,984,389	& 0.99 \\\hline
  458.sjeng	  &   89,612,524	& 88,803,166	& 0.99 \\\hline
  462.libquantum  &   84,762,158	& 84,077,905	& 0.99 \\\hline
  464.h264ref	  &   151,049,588	& 149,284,624	& 0.99 \\\hline
  470.lbm         &   71,878,726	& 71,775,405	& 1.00 \\\hline
  471.omnetpp	  &   382,661,268	& 379,123,833	& 0.99 \\\hline
  473.astar	  &   78,351,117	& 77,872,820	& 0.99 \\\hline
  482.sphinx3	  &   158,434,290	& 156,586,406	& 0.99 \\\hline
  483.xalancbmk	  &   22,485,708,759	& 22,455,060,569& 1.00 \\\hline
    \end{tabular}
  \end{center}
  \caption{Change in total number of instructions executed during the
    compilation with and without \gcm{} (\GCM{}) on \SPEC{} at -Ofast.}
  \label{tab:compile-time}
\end{table}

The compilation time actually reduced for most of the benchmarks as a result of
code motion.  This is because, as \gcm{} removes (congruent) instructions, the
number of instructions to be processed by later compilation passes also
reduces. The compilation time listed here is the total instruction count
executed during the compilation of each benchmark as output from valgrind
-{}-tool=cachegrind \cite{valgrind}.


\begin{table}[h!]
  \begin{center}
    \begin{tabular}{|l|r|r|r|r|r|r|r|r|r|r|r|r|}
      \hline
  &  \multicolumn{3}{|c|}{Hoisted by LICM} &\multicolumn{3}{|c|}{Sunk by LICM} &\multicolumn{3}{|c|}{Functions Inlined} &  \multicolumn{3}{|c|}{Functions Deleted} \\\hline
 Benchmarks       & Base        & \GCM{}	& \%  &Base     & \GCM{}   &\%    & Base	& \GCM{}	  &\%   & Base	& \GCM{}	& \% \\\hline
 400.perlbench	  & 2428	& 2480	& 2.1 & 36	& 35	& -2.8 & 1562	& 1565	  & 0.2 & 232	& 234	& 0.9  \\\hline
 401.bzip2	  & 721	        & 825	& 14.4& 3	& 3	& 0.0  & 172	& 180	  & 4.7 & 32	& 33	& 3.1  \\\hline
 403.gcc	  & 8867	& 10079	& 13.7& 143	& 182	& 27.3 & 5034	& 5080	  & 0.9 & 1303	& 130	& 0.1  \\\hline
 429.mcf	  & 55	        & 51	& -7.3& 1	& 1	& 0.0  & 4	& 4	  & 0.0 & 0 	& 0	& 0.0  \\\hline
 433.milc	  & 2537	& 2795	& 10.2& 3	& 3	& 0.0  & 75	& 81	  & 8.0 & 0	& 0	& 0.0  \\\hline
 444.namd	  & 2204	& 2206	& 0.1 & 240	& 240	& 0.0  & 387	& 390	  & 0.8 & 57	& 55	& -3.5 \\\hline
 445.gobmk	  & 3090	& 3107	& 0.6 & 18	& 19	& 5.6  & 735	& 775	  & 5.4 & 230	& 234	& 1.7  \\\hline
 447.dealII	  & 59114	& 95884	& 62.2& 60	& 8	& -86.7& 100891& 100692  & -0.2& 30219	& 29966	& -0.8 \\\hline
 450.soplex	  & 4668	& 4848	& 3.9 & 18	& 17	& -5.6 & 9801	& 9784	  & -0.2& 3125	& 3112	& -0.4 \\\hline
 453.povray	  & 9533	& 9079	& -4.8& 23	& 32	& 39.1 & 5522	& 5532	  & 0.2 & 1202	& 1201	& -0.1 \\\hline
 456.hmmer	  & 3200	& 3232	& 1.0 & 3	& 8	& 166.7& 562	& 562	  & 0.0 & 103	& 101	& -1.9 \\\hline
 458.sjeng	  & 400	        & 446	& 11.5& 0	& 0	& 0.0  & 127	& 135	  & 6.3 & 18	& 18	& 0.0  \\\hline
 462.libquantum	  & 362	        & 386	& 6.6 & 0 	& 0	& 0.0  & 87	& 87	  & 0.0 & 21	& 21	& 0.0  \\\hline
 464.h264ref	  & 17075	& 17153	& 0.5 & 4	& 4	& 0.0  & 338	& 382	  & 13.0& 72	& 72	& 0.0  \\\hline
 470.lbm	  & 39	        & 39	& 0.0 & 0	& 0	& 0.0  & 13	& 13	  & 0.0 & 4	& 4	& 0.0  \\\hline
 471.omnetpp	  & 565	        & 661	& 17.0& 32	& 30	& -6.3 & 4790	& 4843	  & 1.1 & 1321	& 1316	& -0.4 \\\hline
 473.astar	  & 691	        & 754	& 9.1 & 0	& 0	& 0.0  & 359	& 363	  & 1.1 & 113	& 113	& 0.0  \\\hline
 482.sphinx3	  & 1599	& 1693	& 5.9 & 11	& 9	& -18.2& 198	& 194	  & -2.0& 53	& 53	& 0.0  \\\hline
 483.xalancbmk	  & 9997	& 11453	& 14.6& 56	& 54	& -3.6 & 69534	& 68839	  & -1.0& 29875	& 29519	& -1.2 \\\hline
    \end{tabular}
  \end{center}
  \caption{Static metrics before (Base) and after \gcm{} (\GCM{}) on
    \SPEC{} at -Ofast. The \% Column show percentage increase in metric
    w.r.t. Base. \GCM{} improves LICM and inlining.}
  \label{tab:stats}
\end{table}

Some useful static metrics are presented in Table~\ref{tab:stats}, which shows how \gcm{}
impacted important compiler optimizations like inlining and LICM. LICM removes loop invariant
code out of the loop. It may hoist the code to the loop's preheader or sink the code to loop's
post-dominator \cite{steven1997advanced}. As we can see number of instructions hoisted increased
significantly in several benchmarks, although number of instructions sunk did not change by much.
Number of inlined functions also improved for most benchmarks. In general when more functions
were inlined, more functions were deleted as in 400.perlbench, 445.gobmk.

The compile time metrics of \GCM{} are listed in
Table~\ref{tab:code-motion-metric}. The table lists the number of scalars,
loads, stores and calls hoisted as well as removed. For each category, the
number of instructions removed is greater or equal to the number of instructions
hoisted because each code-motion is performed only when at least one identical
computation is found. Loads are hoisted the most followed by scalars, stores and
calls in decreasing order.  This was the common trend in all our
experiments. One reason why loads are hoisted the most is the early execution of
this pass (before mem2reg pass which scalarizes some memory references) in the
\LLVM{} pass pipeline. Passes like mem2reg and instcombine might actually remove
those loads and the number of hoisted loads may change should the \GCM{} pass be
scheduled later. We can see a few sunk instructions even if \LLVM{} has a code
sinking and hoisting transforms in the SimplifyCFG pass which runs before
\GCM{}: this is because sinking and hoisting in SimplifyCFG have some rather
severe limitations to make the analysis and code transform very fast in order to
be able to run the pass several times.  Another reason is that instructions for
which dependencies are not directly available in the successor
(Section~\ref{subsec:legality-sink}) is not sunk for now. Sinking those
instructions would require a look ahead on the ability to sink the
dependencies. We plan to implement this in near future.

\begin{table}[h!]
  \begin{center}
    \begin{tabular}{|l|r|r|r|r|r|r|r|r|r|}
      \hline
 \GCM{}  &\multicolumn{2}{|c|}{Total Instructions} &\multicolumn{2}{|c|}{Scalars}  & \multicolumn{2}{|c|}{Loads} & \multicolumn{1}{|c|}{Stores} &Calls   &Instructions    \\\hline
 metric           &  H          &R        &H      &R       &H           &R              &H=R    &H & sunk \\\hline
 400.perlbench	  &  2387	& 2456	  &  855  & 857	   & 1493	& 1560		& 38	& 1	& 4	        \\\hline
 401.bzip2	  &  176	& 300	  &  27	  & 27	   & 149	& 273		& 0	& 0 	& 0		\\\hline
 403.gcc	  &  5670	& 5761	  &  2558 & 2567   & 3109	& 3191		& 3	& 0	& 1	        \\\hline
 429.mcf	  &  42	        & 43	  &  5	  & 5	   & 37 	& 38		& 0	& 0 	& 0		\\\hline
 433.milc	  &  128	& 130	  &  10	  & 10	   & 117	& 119		& 0	& 1	& 0		\\\hline
 444.namd	  &  262	& 264	  &  113  & 113	   & 148	& 150		& 0	& 1	& 0		\\\hline
 445.gobmk	  &  1021	& 1052	  &  312  & 314	   & 705	& 734		& 3	& 1	& 1	        \\\hline
 447.dealII	  &  10455	& 12325	  &  4009 & 4063   & 6394	& 8210		& 20	& 32	& 0	        \\\hline
 450.soplex	  &  2389	& 2731	  &  927  & 1004   & 1437	& 1702		& 14	& 11	& 0	        \\\hline
 453.povray	  &  1967	& 2269	  &  560  & 590	   & 1389	& 1661		& 4	& 14	& 0	        \\\hline
 456.hmmer	  &  408	& 427	  &  82	  & 82	   & 319	& 338		& 1	& 6	& 0	        \\\hline
 458.sjeng	  &  372	& 399	  &  128  & 138	   & 234	& 251		& 1	& 9	& 1	        \\\hline
 462.libquantum	  &  49	        & 49	  &  17	  & 17	   & 32	      	& 32		& 0	& 0	& 0		\\\hline
 464.h264ref	  &  2953	& 3438	  &  712  & 779	   & 2238	& 2656		& 2	& 1	& 1	        \\\hline
 470.lbm	  &  23	        & 23	  &  4	  & 4	   & 19		& 19		& 0	& 0	& 0		\\\hline
 471.omnetpp	  &  391	& 408	  &  92	  & 92	   & 297	& 314		& 0	& 2	& 0		\\\hline
 473.astar	  &  202	& 222	  &  62	  & 64	   & 140	& 158		& 0	& 0	& 0		\\\hline
 482.sphinx3	  &  168	& 172	  &  32	  & 32	   & 136	& 140		& 0	& 0	& 0		\\\hline
 483.xalancbmk	  &  5894	& 6162	  &  2586 & 2660   & 3301	& 3495		& 4	& 3	& 2	        \\\hline
 Grand total      &34957	&38631	  &13091  &13418   &21694	&25041	        &90	&82     &10             \\\hline
    \end{tabular}
  \end{center}
  \caption{\GCM{} stats on SPEC2006 at -Ofast, H(Hoisted), R(Removed); Loads are
    hoisted the most followed by scalars, stores and calls.}
  \label{tab:code-motion-metric}
\end{table}

The code shown in Section~\ref{sec:intro} is a reduced example that appears in a
hot loop of a proprietary benchmark.  When the expressions are hoisted from the
conditional clauses, the overall performance of that benchmark improves by
$15\%$ on an \ooo{} processor due to increased instruction level
parallelism, and better scheduling of the instructions, accommodating for the
long latency of the division operation.

\section{Conclusions and Future Work}
\label{sec:future-work}

We have presented the \gcm{} of congruent computations in SSA form. We saw that
it improves inlining and LICM opportunities, reduces register spills, it also
improves performance in some cases. To preserve performance and not hoist/sink
too much, we have implemented a register pressure aware cost model as described
in Section~\ref{subsec:cost-models}. A part of this implementation is already
merged in LLVM trunk as GVNHoist.cpp and we posted the \GCM{} implementation for
review

There are a few improvements that we would like to make. First would be to
integrate this with \LLVM{'}s new GVN-PRE implementation such that more
congruent instructions can be scheduled. This would also expose more
redundancies to the GVN-PRE.  In general it is a good idea to start with lower
ranked expressions first such that maximum hoisting can happen in one iteration,
however, current implementation does not rank the expressions and iteratively
finds a fixed point when no more candidates are available. Even this
implementation converges quickly and no compile time regression have been
observed. This is not the most optimal approach in terms of compile time complexity
and results in multiple data structures to be recomputed. This can be improved
by ranking the computations \cite{rosen1988global}. Also, since \GCM{} runs very
early in the pass pipeline, it will be good to evaluate the code size/performance
impact when it is run in sync with \GVN{}-\PRE{} just like \GCC{} does.

With the implementation of \GCM{} in \LLVM{}, the passes which rely on the
code-size or instruction-counts to make optimization decisions need to be
revisited. The first candidate would be the inliner. We have seen difference
in inlining decisions in Table~\ref{tab:code-size}, when \gcm{}
was enabled.  Since inliner has several cost-models tuned for the previous
pass layout, that would need some improvement.


%% Acknowledgments
\begin{acks}                            %% acks environment is optional
                                        %% contents suppressed with 'anonymous'
  %% Commands \grantsponsor{<sponsorID>}{<name>}{<url>} and
  %% \grantnum[<url>]{<sponsorID>}{<number>} should be used to
  %% acknowledge financial support and will be used by metadata
  %% extraction tools.
We would like to thank Daniel Berlin for his code reviews and for his feedback
on earlier versions of this paper and Brian Grayson for motivating examples that
started this work.
\end{acks}

\newpage

%% Bibliography
\bibliography{Bibliography}

%% Appendix
%\appendix
%\section{Appendix}
%Text of appendix \ldots

\end{document}
